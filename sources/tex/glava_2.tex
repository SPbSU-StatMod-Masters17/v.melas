


\chapter{Функциональный подход}



Настоящая глава содержит изложение идей и результатов разработанного
автором функционального подхода к исследованию локально оптимальных
планов эксперимента для нелинейных по параметрам регрессионных
моделей. Изложение ограничено случаем $D$-критерия оптимальности.
Некоторые другие критерии оптимальности рассматриваются в книге
(Мелас, 1999).

Идея функционального подхода заключается в изучении элементов (точек
и весов) локально оптимальных планов как неявно заданных
вектор-функций параметров, нелинейным образом входящих в
рассматриваемую модель. Эти вектор-функции называются оптимальными
план-функциями (раздел 3.1). В разделе 3.2 вводится основное
векторное уравнение, определяющее оптимальные план-функции. Под
вещественными аналитическими функциями понимаются, как обычно,
вещественные функции, которые в некоторой окрестности любой точки
заданного открытого множества могут быть разложены в (сходящиеся)
ряды Тейлора. В случае, когда функция регрессии является
вещественной аналитической по параметрам и по аргументу, развиваемый
подход позволяет установить условия, при которых оптимальные
план-функции также будут вещественными аналитическими (раздел 3.3).

По существу, результаты этого раздела,~--- глобальная версия
известной теоремы о неявной функции.

В разделе 3.4 вводится представление для матрицы якобиана основного
уравнения, которое позволяет установить невырожденность якобиана для
некоторых классов регрессионных моделей. Результаты этого раздела
ранее не публиковались.

В разделе 3.5 для вычисления коэффициентов рядов Тейлора неявных
функций, вводятся рекуррентные формулы. Эти формулы применимы для
случаев, когда якобиан соответствующей системы уравнений может
обращаться в нуль в исходной точке. Кроме того, эти формулы
ориентированы на использование пакетов символьной обработки данных
Maple и Mathcad. Раздел является изложением работы (Мелас,
Пепелышев, 1999).


\section{Понятие оптимальной план-функции}



Во многих случаях параметры, линейно входящие в исследуемую модель,
не оказывают влияния на вид локально оптимальных планов. В настоящем
разделе мы введем класс регрессионных функций, допускающих
исключение таких параметров. Кроме того, мы введем важное понятие
{\it оптимальной план-функции}, основанное на преобразовании задачи
нахождения оптимальных планов в задачу нахождения экстремума функции
от нескольких переменных (число которых фиксировано).

Предположим, что вектор параметров $\Theta$ имеет вид
$$
\Theta^T=\left(\Theta^T_{(1)}\vdots \Theta^T_{(2)}\right),
$$
где $\Theta^T_{(1)}=(\theta_1,\ldots,\theta_{m-k})$,
$\Theta^T_{(2)}= (\theta_{m-k+1},\ldots,\theta_m)$, причем параметры
$\theta_i$, $i=1,\ldots,m-k$ входят в модель линейным образом, а
параметры $\theta_{i+m-k}$, $i=1,\ldots,k$ входят в модель
нелинейным образом. Пусть $m\geq 2k$. Обозначим $l=m-2k$, $l\geq 0$.
Рассмотрим функцию регрессии вида
$$
\eta(x,\Theta)=\sum^l_{i=1}\theta_i\eta_i(x)+ \sum^{m-
k}_{i=l+1}\theta_i\eta_i(x,\theta_{i+k}).
$$

Пусть
$$
\xi=\{x_1,\ldots,x_n;\mu_1,\ldots,\mu_n\},
$$
где $n$ --- некоторое натуральное число, --- произвольный
(приближенный) план эксперимента.

Тогда определитель информационной матрицы плана $\xi$ принимает вид
$$
\det M(\xi,\Theta)=\theta^2_{l+1}\ldots\theta^2_{m-k}\det \bar
M(\xi,\Theta_{(2)}),
$$
где \bea &&\bar M(\xi,\Theta_{(2)})=\int_{\mbox{\eufrac{X}}}
f(x,\Theta_{(2)}) f^T
(x,\Theta_{(2)}\xi(dx),\\
&&f^T(x,\Theta_{(2)})=(f_1(x),\ldots,f_l(x),f_{l+1}(x,\theta_{m-
k+1}),\ldots,\nonumber \\
&&f_{l+k}(x,\theta_m),f_{l+k+1}(x,\theta_{m-k+1}), \ldots,
f_m(x,\theta_m)),\nonumber \\
&&f_i(x)=\eta_i(x),\quad i=1,\ldots,l, \\
&&f_{l+i}(x,\theta_{m-k+i})=\eta_{l+i}(x,\theta_{m-k+i}),\quad
i=1,\ldots,k, \nonumber \\
&&f_{l+k+i}(x,\theta_{m-k+i})=\frac{\partial}{\partial\theta_{m-k+i}}
\eta_{l+i}(x,\theta_{m-k+i}),\quad i=1,\ldots,k,\nonumber \\
&&\Theta=\Theta_{u}.\nonumber \eea

Пусть $\theta^{(0)}_{l+1},\ldots,\theta^{(0)}_{m-k}\ne 0$. Очевидно,
что локально $D$-оп\-ти\-маль\-ный план зависит только от вектора
$\Theta_{(2)}=\Theta^{(0)}_{(2)}$.

Обозначим $z_1=\theta^{(0)}_{m-k+1},\ldots,z_k=\theta^{(0)}_{m}$.
Пусть $\tilde Z=\tilde Z_{(n)}$ ограниченное замкнутое множество в
$\IR^k$ такое, что при $Z={\rm Int}\,\tilde Z$ для любого $z\in Z$
существует локально $D$-оптимальный план, включающий $n$ различных
точек.

Как показано в разделе 1.3, при $n=m$ веса точек в любом локально
$D$-оптимальном плане имеют вид $\mu_i=1/m$, $i=1,2,\ldots,m$.

Дадим следующие определения.

{\bf Определение 2.1.1.} {\it При $n>m$ оптимальный план-функцией
будем называть вектор-функцию}
$$
\tau^*(z): Z\to V,
$$
где
$$
V=\{\tau=(x_1,\ldots,x_n,\mu_2,\ldots,\mu_n); x_i\in{\mbox{\eufrac
X}},\,\, \mu_i>0,\,\, i=1,\ldots,m,\,\, \sum^n_{i=2}\mu_i<1\}
$$
{\it такую, что при любом фиксированном $z=\Theta^{(0)}_{(2)}\in Z$
точки $x_1^*=\tau^*_1(z),\ldots,x^*_n=\tau^*_n(z)$ и веса
$\mu^*_1=1-\sum^n_{i=2}\mu_i^*$, $\mu^*_i=\tau^*_{n+i}(z)$ образуют
локально $D$-оптимальный план.}

{\bf Определение 2.1.2.} {\it При $n=m$ оптимальной план-функцией
будем называть вектор-функцию}
$$
\tau^*(z)\,:\,Z\to V,
$$
где
$$
V=\{\tau=(x_1,\ldots,x_m);\,\, x_i\in{\mbox{\eufrac X}}\},
$$
{\it такую, что при любом фиксированном $z=\Theta^{(0)}_{(2)}\in Z$
точки $x^*_1=\tau^*_1(z),\ldots,x^*_m=\tau^*_m(z)$ и веса
$\mu^*_i=1/m$, $i=1,\ldots,m$ образуют локально $D$-оптимальный
план.}

Понятие оптимальной план-функции введено в работе (Мелас, 1981) при
изучении нелинейных по параметрам функций регрессии в виде
алгебраической суммы экспонент. Оно является первым шагом к
исследованию зависимости локально оптимальных планов от значений
параметров, нелинейным образом входящих в исследуемую модель.


\section{Основное уравнение}

В настоящем разделе, исходя из необходимых условий экстремума
функции нескольких переменных, мы введем уравнение, определяющее
оптимальные план-функции неявным образом.

Заметим, что при любом фиксированном $z$ вектор $\tau^*(z)$ является
решением задачи
$$
\varphi(\tau,z)\to \sup_{\tau\in V},
$$
где \beq \varphi(\tau,z)=[\det \bar M(\xi,z)]^{1/m}. \label{eq:A73}
\eeq

Рассмотрим сначала частный случай, который нам потребуется в
следующей главе.

Пусть $n=m$, ${\mbox{\eufrac X}}=[r_1,r_2]$, где $r_1< r_2$ ---
произвольные заданные вещественные числа. Так как все точки
$x_i=\tau_i$ по определению различны, то без ограничения общности
будем считать, что
$$
r_1\leq x_1<x_2<\ldots<x_m\leq r_2.
$$

Пусть при $z\in Z$ точка $x_1$ любого локально $D$-оптимального
плана совпадает с $r_1$, а $x_m<r_2$ для такого плана. В этом случае
вектор $\tau$ и функцию $\varphi(\tau,z)$ переопределим следующим
способом: \bea
&&\tau=(\tau_1,\ldots,\tau_{m-1})=(x_2,\ldots,x_m), \nonumber\\
&&\varphi(\tau,z)=[\det \hat M(\zeta,z)]^{1/m}, \nonumber \eea где
$\zeta=(x_1,\ldots,x_m)$, $x_1=r_1$, $x_{i+1}=\tau_i$,
$z=\Theta_{(2)}$,
$$
\hat M(\zeta,z)=\sum^m_{i=1}
f(x_i,\Theta_{(2)})f^T(x_i,\Theta_{(2)}),
$$
функция $f(x,\Theta_{(2)})$ определена формулой (2.2).

В силу принятых предположений при любом фиксированном $z\in\ Z$
максимум функции $\varphi(\tau,z)$, $\tau\in V$ достигается внутри
множества $V$. Поэтому необходимым условием равенства
$\tau=\tau^*(z)$ при любом фиксированном $z\in Z$ является равенство
нулю производных
$$\frac{\partial}{\partial\tau_i}\varphi(\tau,z)=0,\,\, i=1,\ldots,m-
1.
$$

Обозначим
$$
g_i=g_i(\tau,z)=\frac{\partial}{\partial\tau_i}\varphi(\tau,z), \,\,
i=1,\ldots,m-1,
$$
$$
g=(g_1,\ldots,g_{m-1})^T.
$$
Тогда
$$
g(\tau,z)=0
$$
при $\tau=\tau^*(z)$. Это уравнение будем называть {\it основным
уравнением}. Оно позволяет свести задачу к исследованию неявных
функций.

Однако, прежде чем перейти к такому исследованию, рассмотрим более
общий случай. Пусть {\eufrac X}~--- ограниченное замкнутое множество
в $\IR^t$, $t>1$. Предположим, что это множество имеет вид
$$
{\mbox{\eufrac X}}=\{x=(x_{(1)},\ldots,x_{(t)}),\Phi_j(x)\leq 0,\,\,
j=1,2,\ldots,k\},
$$
где $\Phi_j(x)$, $j=1,\ldots,k$~--- некоторые заданные непрерывно
дифференцируемые функции.

Согласно методу неопределенных множителей Лагранжа (Фихтенгольц,
1966, с.\,470) необходимым условием $\tau=\tau^*(z)$ является
равенство нулю производных \bea
\frac{\partial}{\partial\tau_{i(\nu)}}&\left[\varphi(\tau,z)-
\sum^k_{j=1}\sum^n_{i=1} \lambda_{ij}\Phi_j(x_i)\right]&=0,\nonumber\\
\frac{\partial}{\partial\tau_j}&\left[\varphi(\tau,z)- \sum^k_{j=1}
\sum^n_{i=1}\lambda_{ij}\Phi_j(x_i)\right]&=0,\nonumber \eea где
$i=1,\ldots,n$, $\nu=1,\ldots,t$, $j=nt+1,\ldots,2n(t+1)-1$, \bea
\tau&=&(\tau_1,\ldots,\tau_n,\mu_2,\ldots,\mu_n)=
(x_1,\ldots,x_n,\mu_2,\ldots,\mu_n),\nonumber \\
\tau_i&=&(\tau_{i(1)},\ldots,\tau_{i(t)})=
(x_{i(1)},\ldots,x_{i(t)}),\,\, i=1,\ldots,n,\nonumber \eea при
некоторых значениях величин
$$
\{\lambda_{ij}\geq 0; \,\, i=1,\ldots,n,\,\, i=1,\ldots,k\}.
$$
Кроме того, для всех пар $(i,j)\in S=\{(i,j); \lambda_{ij}>0\}$
должно выполняться равенство $\Phi_j(x_i)=0$.

Пусть $r$ число элементов множества $S$  и каждой паре $(i,j)\in S$
сопоставлен номер $\nu=\nu(i,j)$, $\nu=1,2,\ldots,r$.

Обозначим \bea \bar\varphi(\tau,\lambda,z)&=&\varphi(\tau,z)-
\sum^k_{j=1}\sum^n_{i=1} \lambda_{ij}\Phi_j(x_i),\nonumber \\
\lambda_{ij}&=&\lambda_\nu, \,\, \nu=\nu(i,j),\,\,
\nu=1,\ldots,r, \nonumber \\
\lambda_{ij}&=&0,\quad (i,j)\not\in S. \nonumber \eea

Пусть $g=(g_1,\ldots,g_{s})^T$, $s=(t+1)n+r-1$~--- вектор
производных \bea &&\frac{\partial}{\partial\tau_{i(j)}}
\bar\varphi(\tau,\lambda,z),
\,\, i=1,\ldots,n, \,\, j=1,\ldots,t,\nonumber\\
&&\frac{\partial}{\partial\tau_{nt+i}}\bar\varphi(\tau,\lambda,z),
\,\,
i=1,\ldots,n-1,\nonumber\\
&&\frac{\partial}{\partial\lambda_\nu}\bar\varphi(\tau,\lambda,z)=
\Phi_j(x_i), \,\, \nu=1,\ldots,r.\nonumber \eea Тогда необходимым
условием равенства $\tau=\tau^*(z)$ является выполнение векторного
равенства
$$
g(\tau, \lambda, z)=0.
$$

Если при любом фиксированном $z\in Z$ величина $r$ остается
постоянной, то это уравнение определяет вектор-функцию $\tau^*(z)$
неявным образом.

Вернемся к рассмотренному ранее случаю $n=m$, ${\mbox{\eufrac
X}}=[r_1,r_2]$, $x_1=r_1$ для любого локально $D$-оптимального
плана.

Предположим, что функции $\eta_i(x)$, $i=1,2,\ldots,l$,
$\eta_{l+j}(x,z_j)$, $j=1,\ldots,k$~--- вещественные аналитические
функции при $x\in {\rm Int}\,{\mbox{\eufrac X}}$, $z\in Z$. Тогда
функция $\varphi(\tau,z)$, определенная формулой (2.3), является
вещественной аналитической при $z\in Z$, $\tau\in{\rm
Int}\,{\mbox{\eufrac X}}^{m-1}$, так как определитель $\hat
M(\zeta,z)$ представляется в виде алгебраической суммы произведений
функций $\eta_i$ и их производных.

Следовательно, функции
$g_i(\tau,z)=\frac{\partial}{\partial\tau_i}\varphi(\tau,z)$,
$i=1,\ldots,m-1$ являются вещественными аналитическими при $\tau\in
{\rm Int}\,{\mbox{\eufrac X}}^{m-1}$, $z\in Z$.


Очевидно, что такие же выводы можно сделать и в общем случае, если
предположить дополнительно, что функции $\Phi_j(x)$, $j=1,\ldots,k$
являются вещественными аналитическими при $x\in{\rm
Int}\,{\mbox{\eufrac X}}$.

Кроме того предположим, что основное уравнение имеет единственное
решение $\tau=\tau^*(z)$ при любом фиксированном $z\in Z$.


В следующем разделе мы изучим основное уравнение для функций $g$
общего вида, удовлетворяющих сформулированным выше условиям.


\section{Аналитичность неявных функций}

Пусть $m$ и $k$ --- любые натуральные числа, $T$ и $Z$ ---
ограниченные замкнутые множества в ${\IR}^m$ и ${\IR}^k$,
соответственно, $\tilde T={\rm int}\,T$, $\tilde Z={\rm int}\, Z$,
$\hat T$ --- ограниченное замкнутое подмножество $\tilde T$. Пусть
$g(\tau,z)$ --- произвольная вещественная аналитическая
вектор-функция при $\tau\in \tilde T$, $z\in \tilde Z$ такая, что
при любом фиксированном $z\in Z$ уравнение \beq g(\tau,z)=0
\label{eq:AA} \eeq при $\tau\in T$ имеет единственное решение
$\tau=\tau(z)\in\hat T$.

Пусть $\tau^*(z)$ --- вектор-функция, такая, что при фиксированном
$z\in Z$ и $\tau=\tau^*(z)$ выполнено равенство (2.4).

В силу сделанного нами предположения эта функция определена
единственным образом.

Обозначим
$$
J(\tau,z)=\frac{\partial}{\partial\tau}g(\tau,z),
$$
тогда $J(\tau,z)$ есть $m\times m$ матрица.

Предположим, что выполняется условие \noindent
$$
(A)\hskip3cm \det J(\tau^*(z),z)\ne 0\,\,{\rm при}\,\, z\in \tilde
Z.
$$

Имеет место следующая теорема.

\begin{theorem}
\label{th:nef_1} (об аналитичности неявной функции). При
выполнении сформулированных выше условий $\tau^*(z)$ есть
вещественная аналитическая вектор-функция при $z\in \tilde Z$.
\end{theorem}

{\bf Доказательство.}  Пусть $W$ --- любое компактное подмножество
$\tilde Z$, $H=\hat T\times W$, $z_{(0)}\in W$,
$\tau_{(0)}=\tau^*(z_{(0)})$. Тогда согласно нашему допущению
$(\tau_{(0)}, z_{(0)})\in H$.

Рассмотрим всевозможные точки $(\tau_{(0)},z_{(0)})$ такого вида. По
теореме о неявной функции (Бибиков, 1991, с.\,173) для любой такой
точки существует некоторая окрестность $U_{(0)}$, и окрестность
точки $z_{(0)}$, которую обозначим через $W_{(0)}=W(z_{(0)})$,
такие, что при $z\in W_{(0)}$ существует вещественная аналитическая
вектор-функция $\tilde\tau(z)$ со свойствами:
$\tilde\tau(z_{(0)})=\tau_{(0)}$, $(\tau(z),z)\in U_{(0)}$ и
$g(\tilde\tau(z),z)=0$.

В силу единственности решения уравнения (2.4) при любом
фиксированном $z\in \tilde Z$ эта вектор-функция совпадает с
$\tau^*(z)$ при $z\in W_{(0)}$. Окрестности $W_{(0)}=W(z_{(0)})$
образуют покрытие множества $W$. В силу компактности $W$ из этого
покрытия можно выделить конечное подпокрытие, таким образом
$$
W\subset \sum^L_{i=1} W(z^{(i)}_{(0)}).
$$
Следовательно, $\tau^*(z)$ является вещественной аналитической
вектор-функцией на $W$, а значит, и на $\tilde Z$. \hfill$\Box$

В следующем разделе мы получим формулу для матрицы якобиана
уравнения (2.4), которая облегчает проверку условия $(A)$.


\section{Якобиан основного уравнения}

Изучим сначала матрицу якобиана основного уравнения для функций
$\varphi(\tau,z)$ общего вида, представимых в виде минимума
некоторой выпуклой функции.

Пусть $m,k,t$ --- произвольные натуральные числа,
$T\subset\IR^{m-1},Z\subset\IR^k,{\mbox{\eufrac A}}\subset\IR^t$ ---
произвольные открытые множества, причем {\eufrac A} --- выпуклое.

Рассмотрим функцию $q(\tau,a,z),\tau\in T,a\in{\mbox{\eufrac
A}},z\in Z$ со следующими свойствами:

(a) функция $q(\tau,a,z)$ дважды непрерывно дифференцируема по
$\tau$ и $a;$

(б) функция $q(\tau,a,z)$ строго выпукла по $a.$

Пусть, далее, функция $\varphi(\tau,z)$ имеет вид \beq
 \varphi(\tau,z)=\minl_{a\in{\mbox{\eufrac A}}} q(\tau,a,z),
 \label{hi=min_q}
\eeq причем при любых $\tau\in T,z\in Z$ этот минимум достигается.
Из выпуклости функции $q(\tau,a,z)$ по $a$ вытекает, что этот
минимум достигается на единственном векторе
$a=\tilde{a}=\tilde{a}(\tau,z).$ Как известно (Демьянов, Малоземов,
1972) функция минимума дифференцируема, если минимум достигается на
единственном векторе. Поэтому функция $\varphi(\tau,z)$ дважды
непрерывно дифференцируема по $\tau.$ Обозначим
$g(\tau,z)=\pdiff{\tau}\varphi(\tau,z)$ и рассмотрим уравнение \beq
 g(\tau,z)=0.
 \label{eq:g=0}
\eeq Предположим, что это уравнение при любом $z\in Z$ имеет
единственное решение $\tau=\tau^*=\tau^*(z).$ Изучим матрицу \bea
 J(\tau,z)=\frac{\partial}{\partial\tau}g(\tau,z)=
 \left(\frac{\partial^2}{\partial\tau_j\partial\tau_i}\varphi(\tau,z)\right)_{i,j=1}^{m-1}
\eea при $\tau=\tau^*(z).$ Пусть
$a^*=a^*(z)=\tilde{a}(\tau^*(z),z).$

Рассмотрим следующие матрицы \beq \barr{l}
 E=\left(\frac{\partial^2}{\partial\tau_j\partial\tau_i}q(\tau,a,z)\right)_{i,j=1}^{m-1},\\
\\
 B=\left(\frac{\partial^2}{\partial\tau_j\partial a_i}q(\tau,a,z)\right)_{i,j=1}^{t,m-1},\\
\\
 D=\left(\frac{\partial^2}{\partial a_j\partial a_i}q(\tau,a,z)\right)_{i,j=1}^{t}
\earr
 \label{eq:EDB}
\eeq при $\tau=\tau^*,a=a^*.$ Из условия~(б) вытекает, что матрица
$D$ положительно определена и, значит, существует обратная матрица
$D^{-1}.$

\bt \label{th:aem<ab>} При сформулированных выше условиях имеет
место формула
 \bea
  J(\tau^*(z),z)=E-B^TD^{-1}B.
 \eea
\et

\bproof В силу необходимых условий экстремума при любом
фиксированном $z\in Z$ имеет место равенство \bea
 \pdiff{a}q(\tau,a,z)=0
\eea при $\tau=\tau^*=\tau^*(z),a=a^*=a^*(z).$ Рассмотрим это
векторное равенство при фиксированном $z$ и произвольных $\tau,a$
как систему уравнений, неявным образом определяющую функцию
$a(\tau)$ такую, что $a(\tau^*)=a^*.$ Якобиан этой системы в точке
$(\tau^*,a^*)$ равен $\det D\neq 0.$ По теореме о неявной функции
существует единственная непрерывная вектор-функция $a(\tau)$ такая,
что $a(\tau^*)=a^*$ и заданная в некоторой окрестности точки
$\tau^*.$ Эта функция непрерывно дифференцируема и \beq
 \left.\frac{\partial a(\tau)}{\partial \tau}\right|_{\tau=\tau^*}=-D^{-1}B.
 \label{eq:a'=-DB}
\eeq Используя правило дифференцирования сложной функции, получаем
\bea
 \left.\frac{\partial^2}{\partial\tau_j\partial\tau_i}q(\tau,a(\tau),z)\right|_{\tau=\tau^*}&=&
 \left.\frac{\partial^2}{\partial\tau_j\partial\tau_i}q(\tau,a,z)\right|_{\tau=\tau^*,a=a^*}+\nonumber \\
&&\nonumber\\
 &+&\suml_{\nu=1}^t
 \left.\frac{\partial^2}{\partial a_{\nu}\partial\tau_i}q(\tau,a,z)\right|_{\tau=\tau^*,a=a^*}
  \frac{\partial a_{\nu}(\tau^*)}{\partial\tau_j}.\nonumber
\eea Учитывая обозначения~(\ref{eq:EDB}) и
формулу~(\ref{eq:a'=-DB}), получаем \bea
 \left(
 \left.\frac{\partial^2}{\partial\tau_j\partial\tau_i}q(\tau,a(\tau),z)\right|_{\tau=\tau^*}
 \right)_{i,j=1}^{m-1}
 =E-B^TD^{-1}B.
\eea С другой стороны при любом фиксированном $z\in Z$ имеем \bea
 \varphi(\tau,z)=\minl_{a\in{\mbox{\eufrac A}}}q(\tau,a,z)=q(\tau,a(\tau),z)
\eea при $\tau$ из некоторой окрестности точки $\tau^*=\tau^*(z).$
Дифференцируя это равенство дважды по $\tau,$ получаем
 \bea
  J(\tau^*(z),z)=J(\tau^*,z)=E-B^TD^{-1}B.
 \eea
%\etproof

Применим полученный результат к функции $\varphi(\tau,z),$
определенной формулой~(2.3).

Обозначим через $\A$ множество всех положительно определенных
$m\times m$ матриц $A=(a_{ij})$ таких, что $a_{mm}=1.$

\bl
 \label{th:ageom<aa}
 Для любых матриц $A,B\in\A$ выполняется неравенство
 \beq
  (\det B)^{1/m}\le\frac{\tr AB}{m(\det A)^{1/m}}
  \label{eq:ageom<aa}
 \eeq
 причем равенство достигается для единственной матрицы $A\!=\!A^*\!\in\!\A,$
 где $A^*\!=\!const\;\!B^{-1}.$
\el \bproof
 Неравенство~(\ref{eq:ageom<aa}) есть специальный случай неравенства между
 средним арифметическим и средним геометрическим
 (Карлин, Стадден, 1976, c.~325).
%\elproof

\bl
 \label{th:Psi_vipykl}
 Функция $\Psi(B)=(\det B)^{1/m}$ при $B\in\A$ строго вогнутая.
\el \bproof В силу леммы~\ref{th:ageom<aa} имеем \bea
 \Psi(B)=\minl_{A\in\A}\frac{\tr AB}{m(\det A)^{1/m}}.
\eea Следовательно, при $B_1,B_2\in\A$ выполняется \bea
 \Psi((B_1+B_2)/2)=\minl_{A\in\A}\frac{\tr A(B_1+B_2)/2}{m(\det A)^{1/m}}\ge
 \qquad\qquad\qquad\qquad\\
 \ge\frac{1}{2}\left(
 \minl_{A\in\A}\frac{\tr AB_1}{m(\det A)^{1/m}}+
 \minl_{A\in\A}\frac{\tr AB_2}{m(\det A)^{1/m}}
 \right)=(\Psi(B_1)+\Psi(B_2))/2,
\eea причем равенство достигается только при $B_1=B_2.$ %\elproof

\bl
 Функция
 \bea
  \Psi(A)=\frac{\tr AB}{m(\det A)^{1/m}}
 \eea
 при $A\in\A$ и любой матрице $B\in\A$ строго выпуклая.
\el \bproof Заметим, что \beq
 \frac{2}{\alpha+\beta}\le\frac{1}{2}
 \left(\frac{1}{\alpha}+\frac{1}{\beta}\right),
 \label{eq:ab}
\eeq для любых чисел $\alpha,\beta>0.$ В самом деле, это неравенство
эквивалентно неравенству \bea
 (\alpha+\beta)^2-4\alpha\beta=(\alpha-\beta)^2\ge 0.
\eea Теперь утверждение леммы доказывается непосредственным
вычислением с использованием леммы~\ref{th:Psi_vipykl} и
неравенства~(\ref{eq:ab}).

Паре индексов $(i,j),\>i\le j,\>i,j=1,\ldots,m,\>(i,j)\neq(m,m)$
поставим в соответствие число $\nu=\nu(i,j):$ \bea
 \nu(1,1)&=&1,\nu(1,2)=2,\ldots,\nu(2,2)=m+1,\ldots,
 \nu(m-1,m)=\nonumber\\
&=&t=m(m+1)/2-1. \eea Для любого вектора $a\in\IR^t$ определим
матрицу $A(a)$ следующими соотношениями \bea
 a_{ji}=a_{ij}=a_{\nu(i,j)},\>a_{mm}=1,\>i,j=1,\ldots,m,\>i\le j.
\eea Определим множество {\eufrac A} следующим образом \bea
 {\mbox{\eufrac A}}=\{a\in\IR^t:\>A(a)\in\A\}.
\eea Множество {\eufrac A}, очевидно, открыто и выпукло в $\IR^t.$
Определим функцию \bea
 q(\tau,a,z)=(\det A(a))^{-1/m}tr\left(A(a)\ovl{M}(\zeta,z)\right)m,
\eea где матрица $\hat{M}(\zeta,z)$ определена формулой~(2.2).
Рассмотрим функцию $\varphi(\tau,z)=(\det\hat{M}(\zeta,z))^{1/m}.$ В
силу леммы~\ref{th:ageom<aa} для этой функции справедлива
формула~(\ref{hi=min_q}). Все условия, определенные перед теоремой
2.4.1, очевидно, выполнены. Поэтому в силу теоремы 2.4.1. \bea
 J(\tau^*(z),z)=E-B^TD^{-1}B.
\eea Обозначим $\psi(a)=(\det A(a))^{-1/m}.$ Для матриц $B$ и $E$
непосредственным дифференцированием нетрудно проверить
справедливость следующих формул. \bea \barr{l}
 E=diag\{E_{11},\ldots,E_{m-1m-1}\},\\
 E_{ii}=\psi(a^*)\left.\ppdiff{x^2}(f^T(x)A(a^*)f(x))\right|_{x=x^*_{i+1}},
  i=1,\ldots,m-1,\\
 A(a^*)=const\left(\hat{M}(\zeta^*,z)\right)^{-1},\\
 B=(b_{\nu k})_{\nu,k=1}^{t,m-1},\\
 b_{\nu k}=2\psi(a^*)\left.\pdiff{x}(f_i(x)f_j(x))\right|_{x=x^*_k},
  \nu=\nu(i,j).
\earr \eea Заметим, что матрица $J=J(\tau^*(z),z)$ является
отрицательно определенной, а значит и невырожденной, если
выполняется хотя бы одно из следующих условий:

1) все диагональные элементы матрицы $E$ отрицательны;

2) матрица $B$ имеет полный ранг.

В самом деле, матрица $B^TD^{-1}B$ имеет вид $SS^T$ и, значит,
является неотрицательно определенной, а при матрице $B$ полного
ранга --- положительно определенной. Так как $J=E-B^TD^{-1}B,$ то
$J$ отрицательна определена при выполнении любого из условий~1)-2).


\section{Степенные разложения неявных функций}

Известно (Фихтенгольц, 1966, с.~460), что производные неявных
функций могут быть вычислены рекуррентным образом с помощью метода
неопределенных коэффициентов. В данном разделе, следуя работе
(Мелас, Пепелышев, 1999), мы введем рекуррентные формулы,
ориентированные на использование пакетов символьных вычислений Maple
и Mathcad.

Пусть $m$ и $k$ --- произвольные натуральные числа,
$\tau=(\tau_1,\ldots,\tau_m) \in {\IR}^m$, $z=(z_1,\ldots,z_k)\in
{\IR}^k$ и $g(\tau,z)=(g_1(\tau,z), \ldots,g_m(\tau,z))$ ---
вещественная аналитическая вектор-функция в некоторой окрестности
$U$ точки $(\tau_{(0)},z_{(0)})$, такая, что
$g(\tau_{(0)},z_{(0)})=0$.

Для любого набора индексов $s=(s_1,\ldots,s_k)$, $s_i\geq 0$,
$i=1,\ldots,k$ и любой (скалярной, векторной или матричной)
вещественной аналитической функции $f$ обозначим
$$
f_{(s)}=\left(f(z)\right)_{(s)}=\frac{1}{s_1!\ldots s_k!}
\frac{\partial^{s_1}}{\partial z_1^{s_1}}\ldots
\frac{\partial^{s_k}}{\partial z_k^{s_k}} f(z)|_{z=z_{(0)}}.
$$

Обозначим
$$
S_t=\{s=(s_1,\ldots,s_k);\quad s_i\geq 0,\quad \sum^k_{i=1} s_i=t\}
$$
при $t=0,1,\ldots$.

Пусть $l=(l_1,\ldots,l_k)$,$l_i\geq 0$ --- заданный вектор целых
чисел.

Предположим, что функция $g$ имеет вид
$$
g(\tau,z)=(z_1-z_{1(0)})^{l_1}\ldots (z_k-z_{k(0)})^{l_k}
\psi(z)\tilde g(\tau,z),
$$
где $\psi(z)$ --- однородный многочлен степени $p\geq 0$,
$$
\psi(z)=\sum_{s\in S_p}a_{(s)}(z_1-z_{1(0)})^{s_1}\ldots
(z_k-z_{k(0)})^{s_k},
$$
такой, что $a_{(p,0,\ldots,0)}\ne 0$, причем $\det \tilde
J(\tau_{(0)},z_{(0)})\ne 0$, где
$$
\tilde J(\tau,z)=\frac{\partial}{\partial\tau}\tilde g(\tau,z).
$$

Рассмотрим уравнение
$$
g(\tau,z)=0.
$$
При $z_i\ne z_j$, $(i\ne j)$, $z_i\ne z_{i(0)}$, $i,j=1,\ldots,k$ и
при $\psi(z)$ таком, что $\psi(z)\ne 0$ при $z_i\ne z_j$, $i\ne j$,
$z_i\ne z_{i(0)}$, $i,j=1,\ldots,k$, это уравнение эквивалентно
уравнению
$$
\tilde g(\tau,z)=0.
$$

Легко проверить, что функция $\tilde g(\tau,z)$ является
вещественной аналитической при $(\tau,z)\in U$.

В силу теоремы о неявной функции (Бибиков, 1991, с.\,173) существует
вектор-функция $\tau^*(z)$,\, которая\, является\,
ве\-ще\-ст\-вен\-ной аналитической вектор-функцией в некоторой
окрестности $H$ точки $\tau=\tau_{(0)}$, и такая, что
$\tau^*(z_{(0)})=\tau_{(0)}$, $(\tau^*(z),z)\in U$ и $\tilde
g(\tau^*(z),z)=0$, а значит и $ g(\tau^*(z),z)=0$ при $z\in H$. Эта
вектор-функция в некоторой окрестности $H^*\subset H$ точки
$z_{(0)}$ разлагается в (сходящийся) ряд Тейлора:
$$
\tau^*(z)=\tau_{(0)}+ \sum^\infty_{t=1}\sum_{s\in S_t}
\tau^*_{(s)}(z_1- z_{1(0)})^{s_1}\ldots(z_k-z_{k(0)})^{s_k}.
$$
Отметим, что при $k=1$ множество $S_t$ состоит из одного элемента
$s=s_t=t$.

Построим рекуррентные формулы для вычисления коэффициентов
$\tau^*_{(s)}$, $s\in S_t$, $t=1,2,\ldots$.

Пусть $I$\,---\,произвольное\,множество\,индексов
вида\,$s=(s_1,\ldots,s_k)$, $s_i\geq 0$, $i=1,\ldots,k$.

Обозначим
$$
\tau_{<I>}(z)=\sum_{s\in I}\tau_{(s)}(z_1-z_{1(0)})^{s_1}\ldots
(z_k- z_{k(0)})^{s_k},
$$
$$
J(\tau,z)=\frac{\partial}{\partial \tau}g(\tau,z),\quad
J_{(l)}=\left(J(\tau_{(0)},z)\right)_{(l)}.
$$

Пусть сначала $p=0$.

Введем множество индексов
$$
I_n=U^n_{t=0}S_t.
$$

Тогда имеет место следующее утверждение.

{\bf Предложение 2.5.1.} {\it При сформулированных выше условиях
имеют место следующие формулы
$$
\tau^*_{(s)}=-J^{-1}_{(l)}g(\tau^*_{<I>}(z),z)_{(s+l)}
$$
при $s\in S_t$,  $I=I_{t-1}$, $t=1,2,\ldots$ }

Доказательство этого и следующего предложения будет дано в конце
раздела.

С помощью предложения 2.5.1 можно построить рекуррентный алгоритм
вычисления коэффициентов разложения функции $\tau^*(z)$ в ряд
Тейлора при $p=0$ и произвольном $l=(l_1,\ldots,l_k)$, $l_i\geq 0$,
$i=1,\ldots,k$.

{\bf Алгоритм 1.} На шаге $t$ $(t=1,2,\ldots)$ вычисляем все
коэффициенты с индексами из $S_t$ по формуле
$$
\tau_{(s)}=-J^{-1}_{(l)}\left(g(\tau_{<I_{t-1}>}(z),z\right)_{(s+l)}.
$$

Из Предложения 2.5.1 вытекает, что $\tau_{(s)}=\tau^*_{(s)}$, $s\in
S_t$, $t=1,2,\ldots$.

Рассмотрим теперь случай $p>0$. Определим множество индексов
$$
\hat S_t=\{s=(s_1,\ldots,s_k);s_i\geq
0,i=1,\ldots,k,s_1+2\sum^k_{i=2} s_i=t\}.
$$

Пусть $\hat I_n= U^n_{t=0}\hat S_t$, $q=(p,0,\ldots,0)$.

Пусть $J_{(l+q)}=\left(J(\tau_{(0)},z)\right)_{(l+q)}$.


{\bf Предложение 2.5.2.} {\it При $p>0$ имеют место формулы
$$
\tau^*_{(s)}=-J^{-1}_{(l+q)}
g\left(\tau^*_{<I>}(z),z\right)_{(s+l+q)}
$$
при $s\in \hat S_t$, $I=\hat I_{t-1}$, $t=1,2,\ldots$. }

Алгоритм вычисления коэффициентов $\tau^*_{(s)}$ получаем из
Алгоритма 1 заменой множества $S_t$ на $\hat S_t$.


{\bf Алгоритм 2.} {\it При $p>0$ на шаге $t$ $(t=1,2,\ldots)$
вычисляем все коэффициенты с индексами $s\in\hat S_t$ по формуле
$$
\tau_{(s)}=-J^{-1}_{(l+q)} g\left(\tau_{<\hat
I_{t-1}>}(z),z\right)_{(s+l+q)}.
$$
}

Из Предложения 2.5.2 вытекает, что $\tau_{(s)}=\tau^*_{(s)}$, $s\in
\hat S_t$, $t=1,2,\ldots,s$. Отметим, что при $k>1$ в Алгоритме 2 на
каждом шаге вычисляется приблизительно в $k$ раз меньшее число
коэффициентов, чем в Алгоритме 1, а при $k=1$ эти алгоритмы
совпадают.

Перейдем к доказательству предложений  2.5.1 и 2.5.2.

{\bf Доказательство предложения 2.5.1.} Пусть $\tau(z)$ ---
произвольная вещественная аналитическая в некоторой окрестности
точки $z_{(0)}$ вектор- функция. Рассмотрим следующий
вспомогательный результат.

\bl
 \label{th:ageom<cq}
При сформулированных в начале раздела  условиях и при $p=0$, $l=0$
справедливы равенства:
$$
\frac{\partial^n}{\partial z^n}\left[g\left(\tau_{(n)}(z),z\right)-
g\left(\tau(z),z\right)\right]|_{z=z_{(0)}}=0,
$$
при $k=1$ и $\tau_{(n)}(z)=\sum^n_{i=1}\tau_{(i)}z^i+\tau_{(0)}$,
$n=1,2,\ldots$,
$$
\frac{\partial^n}{\partial z^{s_1}_1\ldots\partial z^{s_k}_k} \left[
g\left(\tau_{<I>}(z),z\right)-g\left(\tau(z),z\right)\right]|_{z=z_{(0)}}=0,
$$
при $k\geq 1$, $s\in S_n$, где $I=I_n$, $n=1,2,\ldots$. \el

{\bf Доказательство.} Пусть сначала $k=1$.

Заметим, что справедливо равенство
$$
\frac{\partial}{\partial z}
g\left(\tau(z),z\right)=\frac{\partial}{\partial \tau}
g(\tau,z)|_{\tau=\tau(z)} \times
\tau^{'}(z)+\frac{\partial}{\partial z} g(\tau,z)|_{\tau=\tau(z)}.
$$
В самом деле, по определению производной и в силу формулы
дифференцирования сложной функции, имеем \bea
&&\frac{\partial}{\partial z} g(\tau(z),z)=\lim_{\Delta\to 0}
\frac{g(\tau(z+\Delta),z+\Delta)-g(\tau(z),z)}{\Delta} = \nonumber\\
&&=\frac{\partial}{\partial \tau} g(\tau,z)|_{\tau=\tau(z)}
\tau^{'}(z)+ \frac{\partial}{\partial z} g(\tau,z)|_{\tau=\tau(z)}.
\nonumber \eea

Используя это равенство, получим \bea &&\frac{\partial^n}{\partial
z^n} g(\tau(z),z)= \frac{\partial}{\partial \tau}g(\tau(z),z)
\tau^{(n)}(z) +
\frac{\partial^n}{\partial z^n} g(\tau,z)|_{\tau=\tau(z)}+\nonumber \\
&&+\sum^m_{i,j=1}\frac{\partial^2}{\partial\tau_i\partial\tau_j}
g\left(\tau(z),z\right)\sum^{n-1}_{t=1}\tau^{(t)}_i(z)\tau^{(n-t)}_j(z)
+\ldots+\nonumber\\
&&+\sum^m_{i_1,\ldots,i_n=1}
\frac{\partial^n}{\partial\tau_{i_1}\ldots\partial\tau_{i_n}}
g\left(\tau(z),z\right) \tau^{'}_{i_1}(z)\ldots
\tau^{'}_{i_n}(z).\nonumber \eea

Переходя к пределу при $z\to z_{(0)}$, получаем \bea
&&\frac{\partial^n}{\partial z^n} g(\tau(z),z)|_{z=z_{(0)}}= \nonumber \\
&&=n! J_{(0)}\tau_{(n)}+\frac{\partial^n}{\partial z^n}
g(\tau_{(0)},
z_{(0)})+\ldots+  \\
&&+\sum^m_{i_1,\ldots,i_n=1}
\frac{\partial^n}{\partial\tau_{i_1}\ldots\partial\tau_{i_n}}
g(\tau_{(0)},z_{(0)})\tau_{i_1(1)}\ldots\tau_{i_n(1)} i_1!\ldots
i_n!, \nonumber \eea где правая часть зависит только от
$\tau_{(0)},\ldots,\tau_{(n)}$ и не зависит от
$\tau_{(n+1)},\ldots$. Следовательно,
$$
\frac{\partial^n}{\partial z^n} g\left(\tau(z),z\right)|_{z=z_{(0)}}
= \frac{\partial^n}{\partial z^n}
g\left(\tau_{(n)}(z),z\right)|_{z=z_{(0)}}.
$$

Доказательство леммы для случая $k>1$ проводится аналогичным
образом.

Перейдем к доказательству {\bf Предложения 2.5.1}. Пусть $k=1$,
$l=0$. Заметим, что в правой части равенства (2.28) только первое
слагаемое зависит от $\tau_{(n)}$, а остальные зависят только от
$\tau_t$, $t\leq n-1$. Так как $g(\tau^*(z),z)\equiv 0$ при $z\in
H$, то
$$
-\frac{\partial^n}{\partial z^n} g\left(\tau^*_{(n-1)}(z),
z\right)|_{z=z_0}= n! J_{(0)}\tau^*_{(n)}.
$$
При $k>1$, $l=0$, рассуждения проводятся аналогичным образом.

В случае $l\ne 0$ формулы проверяются непосредственным вычислением
производной с учетом леммы 2.5.1. \hfill$\Box$

Очевидно, что при $p=0$ формулы из предложения 1 остаются верными,
если заменить в них $S_t$ на $\hat S_t$ и $I_t$ на $\hat I_t$.

Рассмотрим теперь случай $p>0$.

{\bf Доказательство Предложения 2.5.2.} Пусть сначала $l=0$.
Заметим, что \beq \left(g(\tau^*_{<I>}(z),z)\right)_{(s+q)}=
\sum_{w+v=s+q}a_{(w)}\tilde g(\tau^*_{<I>}(z),z)_{(v)} \label{eq:2}
\eeq для любого множества индексов $I$.

При $w=q$, единственным вектором $v$, таким, что $w+v=s+q$ является
вектор $s$. Пусть $s \in \hat S_n$, $I= \hat I_{n}$. Заметим, что
при $w\ne q$ любой вектор $v$, такой что $w+v=s+q$, принадлежит
множеству $\hat S_t$, $t\leq n-1$. Отсюда правая часть равенства
(\ref{eq:2}) имеет вид
$$
a_{(q)}\tilde g\left(\tau^*_{<\tilde I_n>}(z),z\right)_{(s)}.
$$
Непосредственным вычислением можно проверить, что $J_{(q)}=a_{(q)}
\tilde J_{(0)}$. Отсюда вытекает справедливость предложения 2.5.2
при $l=0$. Для произвольного $l$ справедливость формул проверяется
непосредственным вычислением. \hfill $\Box$

В работе (Мелас, Пепелышев, 1999) с помощью описанных здесь
алгоритмов построены разложения в ряд Тейлора точек насыщенных
локально $D$-оптимальных планов для функции регрессии вида
$$
\eta(x,\Theta)=\sum^k_{i=1}\theta_i {\mbox{exp}}\,(-\theta_{i+k}x),
$$
где $\Theta=(\theta_1,\ldots,\theta_{2k})^T$, $\theta_i\ne 0$,
$\theta_{i+k}>0$ ($i=1,\ldots,k$), $x\in [0,\infty)$.

В следующей главе алгоритм 1 применяется для дробно-рациональной
функции регрессии.
