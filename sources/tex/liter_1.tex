
\newpage
\centerline{} \vskip2.5cm

\noindent {\bf Литература} \vskip1.5cm
\addcontentsline{toc}{chapter}{Литература {}}
\begin{enumerate}
\item {\it Бибиков Ю.\,H.} Курс обыкновенных дифференциальных
уравнений.  М.: Высшая школа, 1991.

\item {\it Демьянов В.Ф., Малоземов В.H.} (1972). Введение в
минимакс. М.: Hаука.


\item {\it Ермаков С.\,М., Жиглявский А.\,А.} (1987).  Математическая теория
оптимального эксперимента. М.: Наука.

\item{\it Карлин С., Стадден В.} (1976). Чебышевские системы и их при\-ме\-не\-ние
в анализе и статистике. М.: Наука.


\item {\it Мелас В.\,Б.} (1981). Оптимальные планы для экспоненциальной
регрессии~// Математические методы планирования эксперимента~/ Под
ред. В.\,В.\,Пененко. Новосибирск: Наука. C.~174--198.

\item {\it Мелас В.\,Б.} (1999). Общая теория функционального подхода к оптимальному
планированию эксперимента. СПб.: Изд-во С.-Петерб. ун-та.

\item {\it Мелас В.\,Б., Пепелышев А.\,H.} (1999). Степенные
разложения неявных функций и локально оптимальные планы
эксперимента~// Статистические модели с приложениями в эконометрике.
СПб.: Изд-во HИХИ СПбГУ. С.\,108--117.

\item {\it Сеге Г.} (1962). Ортогональные многочлены. М.: Hаука.

\item {\it Фихтенгольц Г.М.} (1966). Курс дифференциального и
интегрального исчисления. Т.\,1. М.: Hаука.

\item {\it JennrichR.\,J.} (1969). Asymptotic properties of non-linear
least squares estimators~// Ann. Math. Stat., 40, 633--643.
\end{enumerate}
