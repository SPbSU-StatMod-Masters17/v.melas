\section{Экспоненциальные модели с двумя параметрами. Построение локально-оптимальных планов}

{\color{blue} Кусок про экспоненциальные модели есть в сборнике (Пененко и т.д.), но несколько  с тем форматом, что был у нас на лекциях}

На протяжении нескольких следующих вопросов мы будем изучать экспоненциальную модель с двумя параметрами.
Пусть $y = \eta(x, \theta) + \varepsilon$, где 
$$\eta(x,\theta) = \sum\limits_{i=1}^{k} a_i e^{0\lambda_ix}, x \in \R_{+}, k \in \mathbb{N}$$

$$M(\xi) = \sum \limits_{i=1}^{n} w_if(x_i)\Tr{f(x_i)}$$
$$\xi_{opt} = \argmax_{\xi} \det M(\xi)$$

Заметим, что в локально-оптимальном плане должно быть по крайнем мере $2k$ точек (иначе ранг матрицы будет меньше $2k$ и определитель будет нулем). 

Как мы покажем позже, функции $f$ образуют систему Чебышева, поэтому есть и верхняя граница на количество точек в оптимальном плане — $\frac{2k(2k+1)}{2}+1$ (Кажется, для систем Чебышева верхняя граница на самом деле $[\frac{2k(2k+1)/2 + 1}{2}]$, но мы это внятно не доказали. Мы получили теорему про $I(c)$, которая дает верхнюю границу для граничных точек (по модулю того, что у нас был конус, а нужно его сечение, но с этим можно бороться). Определитель матрицы будет гармонической функцией\footnote{там всякие попарные произведения, они после оператора лапласса умрут}, поэтому максимум у него на границе (вспоминаем матфизику), так что для $D$-оптимальных планов нам доказывать что-то про внутренность действительно не надо)


Для поиска локальных $D$-оптимальных планов мы будем пользоваться теоремой эквивалентности\footnote{Мы ее формулировали до этого, но пусть будет еще раз}:

\begin{thm}
Пусть $M$ — информационная матрица для параметра $\theta \in \mathbb{\Theta} \subset \R^m$.
Пусть $d(x,\theta) = \Tr{f(x)}M^{-1}f(x) = D(\Tr{f(x)}\hat{\theta})$\footnote{$D(\Tr{f(x)}\hat{\theta}) = E(\Tr{f}\hat{\theta} - \Tr{f}\theta)^2 = \Tr{f}E(\hat{\theta} - \theta)\Tr{(\hat{\theta} - \theta)}f = \Tr{f}D_{\hat{\theta}}f = c\Tr{f}M^{-1}f$, $D_{\hat{\theta}} = \frac{\sigma^2}{n} M^{-1}$}


Эквивалентно:
\begin{itemize}
\item $\xi^{*}$ — $D$-оптимальный план (у нас локально)
\item $\xi^{*}$ — $G$-оптимальный план $\xi^{*} = \argmin_{\xi}\max{x}d(x,\theta, \xi)$ (минимизирует максимальную дисперсию предсказаний)
\item $\max\limits_{x} d(x,\xi^{*}) = m$
\end{itemize}
В опорных точках $D$-оптимального плана  $d(x, \xi^{*}$ принимает свое максимальное значение
\end{thm}

Нам будут интересны специальные типы планов:
\begin{dfn}
План, число точек в котором совпадает с числом параметров, называется насыщенным
\end{dfn}

Для экспоненциальных моделей в большинстве случаев оптимальные планы являются насыщенными. Для дробно-рациональной модели, которую мы будем рассматривать в дальнейшем, все локально D-оптимальные планы будут насыщенными. Отметим важный факт о насыщенных планах:
\begin{thm}
Для насыщенных $D$-оптимальных планов все весовые коэффициенты одинаковы.
\end{thm}

\begin{proof}
$$ M(\xi) = \sum\limits_{i=1}^{m} w_i f(x_i)\Tr{f(x_i)} = FW\Tr{F}$$
$$ \det M(\xi) = \prod\limits_{i=1}^{m}w_i \det F\Tr{F} $$
Видно, что $w_i$ и $F$ можно максимизировать по отдельности. Берем логарифм, вспоминаем правило множителей Лагранжа и получаем, что $w_i = \frac{1}{m}$\footnote{можно и через неравенства между средним геометрическим и средним арифметическим доказать}.
\end{proof}
\begin{note}
Утверждается, что такой план еще и единственный, но откуда это берется не ясно.
\end{note}

Теперь отметим еще один полезный факт, связанный с экспоненциальной регрессией:
$$ \det(M(\xi, a, \lambda)) = C(a) \det\tilde{M(\xi, \lambda)}$$
Таким образом, оптимальный план не зависит от вектора $a$ и можно при поиске плана считать, что $a_i = 1$\footnote{Но нельзя считать, что у нас $k$ параметров, у нас их все равно $2k$, просто при максимизации мы можем считать $a_i=1$, т.к. они на выбор точек плана не влияют.}


Для экспоненциальных систем производные будут образовывать систему Чебышева. 
Производные (с точностью до знака):
$$f_i(x) = e^{-\lambda_ix}, f_{2i} = x e^{-\lambda_i x}$$

Из них получаем множество функций $\{e^{-2\lambda_i x}, e^{-(\lambda_i + \lambda_j) x}, xe^{-(\lambda_i + \lambda_j) x}, x^2e^{-2\lambda_i x} \}$

Этот факт мы доказывали в 4 вопросе.

Перейдем к построению локально-оптимальных планов. Начнем с $k=1$. Тогда
$$\eta(x, \theta) = e^{-\lambda_1x}$$
$$f_1 = e^{-\lambda_1x}$$
$$f_2 = -xe^{-\lambda_1x}$$

\begin{thm}
При $k=1$ существует единственный $D$-оптимальный план
$$ \xi = \begin{pmatrix} 0 & \frac{1}{\lambda_1} \\ \frac{1}{2} & \frac{1}{2}\end{pmatrix}$$
\end{thm}

\begin{proof}
По теорем эквивалентности $d(x, \xi) \leq 2$.
$$\Tr{F} = \begin{pmatrix} f_1(x_1) & f_1(x_2) \\ f_2(x_1) & f_2(x_2)\end{pmatrix}$$
$$ W = \begin{pmatrix} \frac{1}{2} & 0 \\ 0 & \frac{1}{2} \end{pmatrix} $$

$$ \xi = \begin{pmatrix} 0 & \frac{1}{\lambda_1} \\ \frac{1}{2} & \frac{1}{2}\end{pmatrix}$$
{\color{blue} тут надо дописать n простых строчек по обращению матрицы и вычислению $d$. У $d$ получим, что при $x=0$ достигается 2, значит $0$ — точка плана по теореме эквивалентности. Вторую точку можно будет найти, продифференцировав определитель. }

\end{proof}



\begin{thm}
При $k=2$ существует единственный $D$-оптимальный план. Кроме того, этот план будет насыщенным.
\end{thm}


\begin{proof}
{\color{blue} TODO: Тут еще больше вычислений и начинают использоваться системы Чебышева}
\end{proof}
