\section{Насыщенные локально D-оптимальные планы для экспоненциальных моделей с двумя параметрами}

В данном разделе мы будем изучать насыщенные локально $D$-оптимальные планы. Наша цель — доказать, что существует единственное насыщенный локально $D$-оптимальный план.

Производные данных моделей имеют вид:
$\{e^{-\lambda_i x}, -a_i x e^{-\lambda_i x} \}$
$$M(\xi, a, \lambda) = \int f(x)\Tr{f(x)}d\xi(x)$$

{ \color{blue} Объяснить через Бине-Коши, почему можно $a_i = 1$ взять }

Пусть
\begin{equation}
\begin{split}
 &F = (f(x_1), …, f(x_n)) \\
 & W = \diag(w_1, …,w_n) \\
 & \Tr{F} = \begin{pmatrix} \Tr{f(x_1)} \\ … \\ \Tr{f(x_n)} \end{pmatrix} 
 \end{split}
 \end{equation}

 Из последнего разложения видно, что определитель $\det M(\xi)$ можно посчитать по форумуле Бине-Коши.  
 Это позволит нам показать, что максимум определителя (по точкам $x_1, …, x_n$) не зависит от значений параметра $a$.

 Мы будем изучать насыщенные планы. В таком случае число точек плана совпадает с числом параметров и задачи максимизации определителя $\det M(\xi)$ становится эквивалентна задачи максимизации $(\det F)2$\footnote{Перенести сюда док-во того, что в насыщенном плане все веса одинаковы.}. Для экспоненциальных система матрица $F$ имеет следующий вид:
 \begin{equation}
\Tr{F} = \begin{pmatrix} e^{-\lambda_1 x_1} & e^{-\lambda_1 x_2}   & … & e^{-\lambda_1 x_{2k}} \\
e^{-\lambda_2 x_1} & e^{-\lambda_2 x_2}   & … & e^{-\lambda_2 x_{2k}} \\ 
… & … & … & …\\
e^{-\lambda_k x_1} & e^{-\lambda_k x_2}   & … & e^{-\lambda_k x_{2k}} \\ 
-xe^{-\lambda_1 x_1} & -xe^{-\lambda_1 x_2}   & … & -xe^{-\lambda_1 x_{2k}} \\ 
… & … & … & …\\
-xe^{-\lambda_k x_1} & -xe^{-\lambda_k x_2}   & … & -xe^{-\lambda_k x_{2k}} \\ 
\end{pmatrix}
 \end{equation}
 Соответсвенно нам надо максимизировать $|\det \Tr{F}|$. Можно показать, что этот определитель совпадает с определителем матрицы\footnote{мы просто вынесли знак из столбцов с $x$ + воспользовались тем, что когда расскладывается по формуле Бине-Коши получается, что необходимо искать максимум  $(-1)^k\det F$, соответсвенно $(-1)^k$ сокращается с вынесенными из столбцов минусами.}
 \begin{equation}
 M(\xi, \lambda) = \Tr{\begin{pmatrix} e^{-\lambda_1 x_1} & e^{-\lambda_1 x_2}   & … & e^{-\lambda_1 x_{2k}} \\
e^{-\lambda_2 x_1} & e^{-\lambda_2 x_2}   & … & e^{-\lambda_2 x_{2k}} \\ 
… & … & … & …\\
e^{-\lambda_k x_1} & e^{-\lambda_k x_2}   & … & e^{-\lambda_k x_{2k}} \\ 
xe^{-\lambda_1 x_1} & xe^{-\lambda_1 x_2}   & … & xe^{-\lambda_1 x_{2k}} \\ 
… & … & … & …\\
xe^{-\lambda_k x_1} & xe^{-\lambda_k x_2}   & … & xe^{-\lambda_k x_{2k}} \\ 
\end{pmatrix}}
\end{equation}

Таким образом, мы решаем задачу
\begin{equation}
\label{maxEqExp}
xi^{*}(\lambda) = \argmax\limits{xi} \det M(\xi, \lambda)
\end{equation}
{\color{red}
Внимание! Чтобы в дальнейшем не вводить много новых обозначений, под $M(\xi, \lambda)$ будем понимать не саму матрицу, а ее определитель. Если будем необходима указать на матрицу, это будет явно сказано. Кроме того, не забывайте, что  выписана у нас не информационная матрица, а некоторый ее эквивалент с точки зрения $D$-оптимальности
}

Не умаляя общности будем считать, что $\lambda_1 > \lambda_2 > … > \lambda_k$.  Наша цель — заменить \eqref{maxEqExp} решение некоторого уравнения, имеющее единственное решение при любом  $\lambda$. В таком случае, при введенных на $\lambda$ ограничениях, решение также будет единственно.

Для этого нам потребуется несколько вспомогательных результатов.
\begin{lem}
Справедлива следующая формула
\begin{equation}
\begin{split}
&M(\xi, \lambda) =   \frac{1}{2! … (2k-1)!} e^{-d \sum\limits_{i=1}^{2k} x_i} \prod \limits_{1 \leq p < q \leq l} (\lambda_p - \lambda_q)^4 \prod \limits_{2k \geq i > l \geq 1} (x_i - x_j) \\
&\{ 1 - \frac{1}{2k} \left(\sum \limits_{j=1}^k( \lambda_j- d)\right) \left(\sum x_i\right) 
+ \frac{1}{2}\sum \limits_{j=1}^{k} (\lambda_j - d)^2 \frac{\sum x_i^2 (2k-1) - 2\sum \limits_{i\neq j} x_i x_j}{2k(2k-1)(2k+1)} + \\
&\frac{1}{2} \left( \sum (\lambda_j - d) \right) 
\frac{\sum x_i ^ 2 (2k-1) + 4k \sum\limits_{i \neq j} x_i x_j}{2k (2k-1)(2k+1)} + o(\alpha)\}
\end{split}
\end{equation}
где $\alpha = \max\limits_{i}(\lambda_i - d )^2$, $d$ произвольное вещественное число, $\frac{o(\alpha}{\alpha} \rightarrow 0$ при $\alpha \rightarrow 0$.
\end{lem}
\begin{proof}
Написать док-во
\end{proof}

Теперь докажем следующую вспомогательную лемму
\begin{lem}
\begin{itemize}
\item
Имеет место равенство
\begin{equation}
\lim \limits_{\lambda \rightarrow \lambda_d} \frac{M(\xi, \lambda)}{\prod\limits_{i < j}(\lambda_i - \lambda_j)^4} = \frac{1}{2!…(2k-1)!}\exp\left(-d\sum\limits_{i=1}^{2k}x_i\right)\prod\limits_{1 \leq i < j \leq 2k}(x_j - x_i)
\end{equation}
где $\lambda_d = \Tr{(d, d,…,d)}$, а $d$ является произвольным вещественным числом.

\item Определим функцию $\tilde{M}$
\begin{equation}
\tilde{M}(\xi, \lambda) = \frac{M(\xi, \lambda)}{\prod \limits_{i < j} (\lambda_i - \lambda_j)^4}
\end{equation}
Тогда функцию $\tilde{M}$ можно доопределить по непрерывности в точках $\lambda$  таких, что для некоторых $i$, $j$ будет выполнено $\lambda_i = \lambda_j$ 
\end{itemize}
\end{lem}
\begin{proof}
Первое утверждение следует из предыдущей леммы.

Написать док-во второго утверждения
\end{proof}


Таким образом мы можем рассматривать $\tilde{M}(\xi, \lambda)$ как непрерывную функцию на множестве $\Lambda = \{ \lambda \in \R^{k} | \lambda_i > 0\}$. 
\begin{dfn}
Оптимальной план-функцией будем называть отображение 
$$\xi^{*}(\lambda): \Lambda \rightarrow \mathbb{X}^{2k}, \mathbb{X} = (0, \infty)$$
удовлетворяющее при любом $\lambda \in \Lambda$
$$\tilde{\xi}^{*}(\lambda) = \max\limits_{\xi}\tilde{M}(\xi, \lambda)$$
\end{dfn}
Заметим, что оптимальная план-функция при любом фиксированном $\lambda \in \Lambda$  задает насыщенный локально $D$-оптимальный план. Как мы покажем далее, эта функция является единственной и локально $D$-оптимальный план определен единственным образом.

Далее из док-ва леммы следует, что 
$$\frac{\partial}{\partial x_1} \tilde{M}(\xi, \lambda) \leq 0$$
А значит точка $x_1 > 0$ не может быть точкой локального экстремума функции $\tilde{M}(\xi, \lambda)$, а значит $x_1^{*}(\lambda) = 0$. 

Нам потребуется еще один вспомогательный результат
\begin{lem}
Значение $\xi^{*}(\lambda)$ в точке $\lambda_d$ ($\lambda_d = \Tr{(d, d,…,d)}, d > 0$)  определено единственным образом и 
$\xi(\lambda_d) = (0, \gamma_2, …, \gamma_{2k})$, где $\gamma_i = \frac{2\gamma_{i-1}'}{d}$. Точки $\{ \gamma_i' \}_{i=1}^{2k-1}$ являются корнями многочлена Лагерра с параметром $\alpha = 1$ $L_{2k-1}^1(x)$.  
\end{lem}
\begin{proof}
Без доказательства
\end{proof}

\subsection{Структура матрицы Якоби}
Мы хотим свести задачу поиска оптимального плана к задаче поиска неявной функции. Предыдущие леммы позволяют нам сформулировать задачу поиска оптимального плана как решение следующей системы:
\begin{equation}
\label{functionalExpModel}
\begin{split}
&x_1^{*}(\lambda)=0 \\
&\frac{\partial}{\partial x_i}\tilde{M}(\xi^{*}(\lambda), \lambda) = 0, i = 2…2k \\
& 0 < x_2^{*}(\lambda) < … < x_2k^{*}(\lambda)\\
\end{split}
\end{equation}
Строгое неравенство для точек плана следует из того, что если есть хотя бы две совпадающие, то определитель (которому равносильна $\tilde{M}$) обращается в ноль. 

Полученная система задает оптимальный план $\xi^{*}(\lambda)$ неявным образом. Для того, чтобы воспользоваться теоремой о неявной функции нам нужны некоторые хорошие свойства от якобиана данной системы. Более того, для того, чтобы план был локальным максимумом целевой функции необходимо, чтобы  
якобиан был отрицательно-определенным. Для рассматриваемой нами функции это будет действительно верно:
\begin{thm}
Матрица $\{ \frac{\partial^2}{\partial x_i \partial x_j} \tilde{M}(\xi^{*}(\lambda), \lambda)\}$, где $\xi^{*}$ удовлетворяет \eqref{functionalExpModel}, а  $\lambda \in \Lambda$ является отрицательно определенной.
\end{thm}
\begin{proof}
написать док-во
\end{proof}

Благодаря этой теореме мы можем применить теорему о неявной функции. 
Для любой точки $\lambda \in \Lambda$ существует окрестность $W(\lambda)$  в которой \eqref{functionalExpModel} $\xi^{*}$ задает неявную функцию 
$\xi^{*}(\lambda)=(0, x_2^{*}(\lambda),…)$, т.е.  
$$\xi^{*} = \xi^{*}(\lambda) \Leftrightarrow \frac{\partial}{\partial x_i}\tilde{M}(\xi^{*}(\lambda), \lambda)=0$$
Теорема о неявной функции также дает нам, что $\xi^{*}$ непрерывна и определена единственным образом в данной окрестности.


\subsection{Теорема о единственности насыщенных локально D-оптимальных планов для экспоненциальных моделей}
Теорема о неявной функции дает единственность только локально (в окресности каждой точки $(x_1, …, x_{2k}, \lambda_1,…,\lambda_{k})$). Докажем теперь, что единственность будет глобальной.

\begin{lem}
Пусть $W$ — окрестность точки, в которой определена $\xi^{*}(\lambda)$. Тогда $x_{i}^{*}(\lambda)$ для $i=2…2k$ строго-моннотоно убывает по каждому $\lambda_{j}$ ($\lambda \in W$).
\end{lem}

{ \color{blue} тут надо доказать факт про то, что функцию $\xi^{*}(\lambda)$ можно доопределить на множестве $\Lambda_{\lambda'} = \{\lambda \in \R^k | \lambda_i \geq \lambda' \}$, если $\xi^{*}$ была изначально задана в окрестности $\lambda'$ }
\begin{lem}
Решение системы \eqref{functionalExpModel} единственно для любого $\lambda \in \Lambda$.
\end{lem}
\begin{proof}
Пусть это не так. Значит существует $\lambda'$ и $\xi_1$ и $\xi_2$ такие, что 
$$\frac{\partial}{\partial x_i} \tilde{M}(\xi, \lambda') = 0 (i = 2…2k)$$
где последнее выражение верно для $\xi = \xi_1$ и $\xi=\xi_2$.
За счет предыдущих лемм можно определить $\xi_1(\lambda)$ и $\xi_2(\lambda)$ — непрерывные вектор-функции на множестве $\Lambda = \{ \lambda: \lambda_i \geq\lambda_i'\}$ такие, что
$$\xi_1(\lambda')=\xi_1$$
$$\xi_2(\lambda')=\xi_2$$
$$\frac{\partial}{\partial x_i} \tilde{M}(\xi_i, \lambda) = 0$$
и $\xi_{i}(\lambda)$ задает план ($\xi = (0, x_2, …, x_{2k}$, $0 < x_2 < … < x_{2k}$).
Теперь рассмотрим $\lambda_d = (d, …, d)$, где $d = \max(\lambda_{i}')$. В этой точке функции совпадают. Кроме того, они непрерывны. Следовательно существует точка $\tilde{\lambda}$ такая, что в ее окрестности заданы две несовпадающие вектор-функции $\xi_1(\lambda)$ и $\xi_2(\lambda)$, а в самой точке $\xi_1(\tilde{\lambda}) = \xi_2(\tilde{\lambda})$. Получаем противоречие с единственностью из теоремы о неявном отображении.
\end{proof}

Также верна следующая лемма:
\begin{lem}
Функции $\xi_i^{*}(\lambda)$ являются аналитическими при $\lambda \in \Lambda$
\end{lem}

В итоге мы пришли к следующей теореме:
\begin{thm}
Оптимальная план-функция существует и определена единственным образом. Первая точка плана $x_1$ находится в нуле, поэтому ее можно рассматривать как функцию $\xi{\lambda} : \Lambda \rightarrow [0, \infty)^{2k-1}$. Кроме того, координатные функции являются аналитическими и строго убывают по каждому $\lambda_j$. План $\xi(\lambda)$ является насыщенным $D$-оптимальным при любом фиксированном $\lambda: \lambda_1 > \lambda_2 … > \lambda_k$.
\end{thm}



