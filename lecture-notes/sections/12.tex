\section{Основное уравнение функционального подхода. Теорема о неявной функции}
\label{funcApproach}
Начнем с теоремы о неявной функции. 
\begin{thm}
Пусть задана функция $q(\tau,z): \R^{s+k} \rightarrow \R$ и пусть $q$ — непрерывно-дифференцируема в окрестности $U \subset \R^{s+k}$. Пусть в точке $(\tau^0, z^0)$ выполнено:
\begin{enumerate}
\item $q(\tau^0, z^0) = 0$
\item $\det J \neq 0$, где $J = \frac{\partial q}{\partial z_i}|_{(\tau^0, z^0)}$
\end{enumerate}

Тогда в некоторой окрестности $W\subset U$ $q$ задает неявную функцию, т.е. существует и единственна такая $\tau=\tau(z)$, что
$q(\tau, z) = 0 \Leftrightarrow \tau = \tau(z)$.

Более того, если $q$ — вещественно-аналитическая\footnote{т.е. в окрестности любой точки расскладывается в ряд Тейлора (сходящийся)}, то и $\tau(z)$ также будет вещественно-аналитической функцией\footnote{Если быть более точным, то гладкость $\tau(z)$ совпадает с гладкостью $q$}.
\end{thm}

Эта теорема нам интересна для решения следующей задачи. Как обычно, хочется найти такой план $\xi$, что $M(\xi, \theta)$ будет в некотором смысле большой матрицей. Мы под «большой» в данном разделе будем понимать $D$-оптимальной:
\begin{equation}
\label{optdesign}
\xi = \argmax\limits_{\xi} \det M(\xi, \theta)
\end{equation}

Не умаляя общности, будем  считать, что все параметры у нас входят нелинейно\footnote{Можно показать, что определитель не зависит от линейно-входящих параметров (смотри пособие)}. Кроме того, введем еще несколько упрощений:
\begin{enumerate}
\item Пусть мы ищем насыщенный план (т.е. количество точек в нем совпадает с кол-вом параметров, следовательно, веса у всех точек плана одинаковы). Тогда план задается с помощью $m$ элементов $x_1, …, x_m$ множества $\mathbb{X}$.
\item  Будем считать, что $\mathcal {X} \subset \R^k$ и любой оптимальный план является внутренней точкой $\mathcal {X}$ (хотим написать достаточное условие минимума). 
\end{enumerate}
 Тогда для решение задачи \eqref{optdesign} при фиксированом $\theta$  можно взять производные и приравнять их к нулю\footnote{Получим необходимое условие максимума. Хорошо бы еще проверить, что якобиан будет отрицательно-определен, да и производные мы можем брать, но кого это волнует…}

\begin{equation}
g_i(\xi, \theta) = \frac{\partial{}}{\partial{\xi_i}} \det M(\xi, \theta) = 0
\end{equation} 

Получаем уравнение:
\[g(\xi, \theta) = 0\]
решениями которого являются $\xi = \xi(\theta)$ — локально D-оптимальные планы.  Это уравнение будем называть основным уравнением функционального подхода. %Если же сделанные нами предположения не верны, то для получения основного уравнения требуется 
 Если не делать предположений о том, что решения — внутренние точки и планы насыщенные, то для получения основного уравнения требуется использовать множители Лагранжа\footnote{а если быть еще более точным, то теорему Куна-Такера \url{https://en.wikipedia.org/wiki/Karush–Kuhn–Tucker_conditions}}.
