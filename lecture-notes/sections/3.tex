 \section{Системы Чебышева. Два эквивалентных определения.}

 \begin{dfn}
Пусть $u_0, …, u_n$ — заданные непрерывные функции на $[a,b]$.  Система называется системой функций Чебышева, если 
определители
$$U\left(
\begin{array}{cccc}
0 & 1 & … & n \\
t_0 & t_1 & … & t_n\\
\end{array}  \right) = det \left(
\begin{array}{ccc}
u_0(t_0) & … & u_0(t_n) \\
u_1(t_0) & … & u_1(t_n) \\
… & … & … \\
u_n(t_0) & … & u_n(t_n) \\
\end{array}
 \right)
 $$
 строго положительны для $\forall a \leq t_0 < t_1 < … < t_n \leq b$
 \end{dfn}
 \begin{dfn}
Обобщенным многочленом называется функция $u(t) = \sum \limits_{i=0}^n a_iu_i(t), a_i \in \R$
 \end{dfn}

\begin{dfn}
Многочлен называется нетривиальным, если $\sum \limits_{i=0}^n a_i^2 \neq 0$
\end{dfn}

\begin{dfn}
\label{chebSystem2}
Система функций называется системой Чебышева, елси любой нетривиальный обобщенный многочлен имеет не более $n$ нулей
\end{dfn}



\begin{thm}
\label{chebDefEqual}

\end{thm}