 \section{Системы Чебышева. Два эквивалентных определения.}

\begin{dfn}[Конструктивное]
    \label{chebSystem1}
    Пусть $u_0, …, u_n$ — заданные непрерывные вещественные функции на $[a,b]$.  Система называется системой функций Чебышева, если 
    определители
    $$U\left(
    \begin{array}{cccc}
        0 & 1 & … & n \\
        t_0 & t_1 & … & t_n\\
    \end{array}  \right) = \det \left(
    \begin{array}{ccc}
        u_0(t_0) & … & u_0(t_n) \\
        u_1(t_0) & … & u_1(t_n) \\
        … & … & … \\
        u_n(t_0) & … & u_n(t_n) \\
    \end{array}
    \right)
    $$
    строго положительны для $\forall a \leq t_0 < t_1 < … < t_n \leq b$.\footnote{На самом деле, ничего ведь страшного, если
    все определители будут строго отрицательны? Это используется в теореме этого билета, обратите на это внимание.}
\end{dfn}
{\color{blue} Здесь нужно рассказать о (по всей видимости) естественности такой штуки через определитель Вандермонда, но я пока сам не понимаю.}
\begin{dfn}
    Обобщенным многочленом называется функция $u(t) = \sum \limits_{i=0}^n a_iu_i(t), a_i \in \R$.
    \footnote{Здесь не накладывается никаких дополнительных ограничений! Просто произвольная линейная комбинация.}
\end{dfn}

\begin{dfn}
    Многочлен называется нетривиальным, если $\sum \limits_{i=0}^n a_i^2 \neq 0$.
    {\color{blue} \textbf{Придирка:} Это условие глядится странновато. На $u_i$ задана упорядоченность или нет?
        Если да, значит обобщенные многочлены не просто так названы многочленами.
        У любого нормального многочлена есть степень! Тут она тоже должна быть, иначе термин обобщенный многочлен слишком натянут.
    А если есть степень, то разумно требовать, чтобы коэффициент при старшем члене был не 0.}
\end{dfn}

Количество нулей обобщенного многочлена $u$ обозначим $Z(u)$.
\begin{dfn}[Аксиоматическое]
    \label{chebSystem2}
    Система вещественных, непрерывных функций $\{u_i\}_{i=0}^n$, определенных на отрезке $[a, b]$ называется системой Чебышева
    если $Z(u) \leqslant n$ для любого нетривиального обобщенного многочлена $u$, построенного по этой системе.
\end{dfn}



\begin{thm}
    \label{chebDefEqual}
    Пусть $\{u_i\}_{i=0}^n$ --- система вещественных непрерывных функций, определенных на отрезке $[a, b]$.
    СУР:
    \begin{enumerate}
        \item Система $\{u_i\}_{i=0}^n$ с точностью до знака некоторых из $u_i$\footnote{\color{blue} Наверное
            это нужно написать формально, но мне не приходят в голову изящные способы}  образует систему Чебышева \ref{chebSystem1}.
        \item Система $\{u_i\}_{i=0}^n$ образует систему Чебышева \ref{chebSystem2}.
    \end{enumerate}
\end{thm}
\begin{proof}
    Пусть $a = \Tr{(a_0, \ldots, a_n)} \in \R^{n + 1}$ такой, что $\sum_{i=1}^n a_i^2 \neq0$..
    Рассмотрим обобщенный многочлен $u(t) = \sum_{i=0}^n a_i u_i(t)$.
    Для произвольного набора точек $\{t_i\}_{i=0}^n \subset [a, b]$ введем матрицу
    \begin{gather*}
        \mathbf U(t_0, t_1, \ldots, t_n) =
        \begin{pmatrix}
            u_0(t_0) & \ldots & u_0(t_n) \\
            u_1(t_0) & \ldots & u_1(t_n) \\
            \ldots & \ldots & \ldots \\
            u_n(t_0) & \ldots & u_n(t_n)
        \end{pmatrix}.
    \end{gather*}

    $1 \rightarrow 2$.
        Пусть $Z(u) \geqslant n + 1$ и $t_0, t_1, \ldots, t_n$ --- первые $n + 1$ нулей многочлена $u$.
        Тогда $\mathbf U(t_0, t_1, \ldots, t_n) \, a = \mathbf 0$\footnote{\color{blue} Здесь времененный шрифт.},
        что противоречит невырожденности $\mathbf U$.

    $2 \rightarrow 1$.
        Пусть система $\{u_i\}_{i=0}^n$ --- не чебышевская в смысле определения \ref{chebSystem1}.
        Тогда найдется такой набор точек $t_0, t_1, \ldots, t_n$,
        матрица $\mathbf U = \mathbf U(t_0, t_1, \ldots, t_n)$ и
        вектор $a = \Tr{(a_0, a_1 \ldots, a_n)} \in \R^{n + 1}$, что $\mathbf U a = \mathbf 0$.
        То есть существует обобщенный многочлен, имеющий не менее $n + 1$ нуля. Противоречие.
\end{proof}
