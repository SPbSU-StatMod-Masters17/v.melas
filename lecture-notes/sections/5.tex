\section{Расширенные системы Чебышева}

Основная цель данного вопроса --- расширить определение систем Чебышева, таким образом, чтобы с помощью
них можно было бы выразить некоторые простые условия, на входящие в эту систему функции.

Запишем на языке чебышевских систем простое условие строгого возрастания функции.
Пусть задана система из двух функций: $u_0(t) = 1$, $u_1(t)$ для $t \in [a, b]$.
Условием того, что эта система будет чебышевской является следующее условие на определитель:

\begin{gather*}
    \begin{vmatrix}
        1 & 1 \\
        u_1(t_0) & u_1(t_1)
    \end{vmatrix} 
    > 0,
\end{gather*}
для $a \leqslant t_0 < t_1 \leqslant b$.

Условие строго возрастания естественно ослабляется до нестрогой монотонности.
С другой стороны, условие строгого возрастания естественно усиливается существованием
строго возрастающей производной. Запишем эти два условия с точки зрения определителей.
Условие нестрого возрастания:

\begin{gather}
    \label{sec5::eq::strong_ar}
    \begin{vmatrix}
        1 & 1 \\
        u_1(t_0) & u_1(t_1)
    \end{vmatrix} 
    \geqslant 0,
\end{gather}
для $a \leqslant t_1 < t_2 \leqslant b$.

Условие строгой монотонности производной записывается следующим образом:
\begin{gather}
    \label{sec5::eq::strong_ar_d}
    \begin{vmatrix}
        1 & u_0'(t) \\
        u_1(t) & u_1'(t)
    \end{vmatrix} = u_1'(t) > 0, \end{gather} для $a \leqslant t \leqslant b$.

Перейдем к обобщению понятия системы Чебышева на произвольное количество функций
так, чтобы
оно описывало условие нестрогой монотонности функции и строгой монотонности производной.

Ясно, что \eqref{sec5::eq::strong_ar} обобщается на произвольное количество функций:
\begin{dfn}
    \label{dfn::ChebSystemWeak}
    Система вещественных, непрерывных функций $\{u_i\}_{i=1}^n$, заданных на отрезке $[a, b]$,
    называется слабой системой Чебышева, если определители
    $$U\left(
    \begin{array}{cccc}
        0 & 1 & … & n \\
        t_0 & t_1 & … & t_n\\
    \end{array}  \right) = \det \left(
    \begin{array}{ccc}
        u_0(t_0) & … & u_0(t_n) \\
        u_1(t_0) & … & u_1(t_n) \\
        … & … & … \\
        u_n(t_0) & … & u_n(t_n) \\
    \end{array}
    \right) \geqslant 0
    $$
    для $a \leqslant t_0 < t_1 < … < t_n \leqslant b$.
\end{dfn}

Посмотрим теперь на второе условие \eqref{sec5::eq::strong_ar_d}.
Заметим, что это условие отличается от обычного определения системы Чебышева тем, что в нем допускается ``совпадение точек'' $t_i = t$.
Точнее говоря, определители в \ref{chebSystem1} (и в \ref{dfn::ChebSystemWeak})  считаются
в строго различных точках: $\{t_i\}_{i=0}^n$: $a \leqslant t_0 < t_1 < … < t_n \leqslant b$, а в \eqref{sec5::eq::strong_ar_d}
определитель вычисляется в некоторой заданной точке $t \in [a, b]$.
Таким образом, необходимо сконструировать такое обобщение стандартного определения \ref{chebSystem1}, которое бы допускало
равенство точек $\{t_i\}_{i=0}^n$.

{\color{blue} В книге Карлина и Штаддена на страницах 16-18 приводится более общее изложение данного материала, я постарался
сделать более элементарное.}

Начнем с некоторого интуитивного понимания идеи. Для начала будем считать, что функции $\{u_i\}_{i=0}^n$ достаточное число раз
дифференцируемы на интервале $(a, b)$.
Рассмотрим некоторый набор точек $\{t_i\}_{i=0}^n$ таких, что $t_0 \leqslant t_1 \leqslant \ldots \leqslant t_n$.
Пусть теперь $t = t_i = t_{i+1} = \ldots = t_{i + q} \not \in \{a, b\}$\footnote{Это чисто формальное условие. Зачем оно нужно?},
где $0 \leqslant i < i + q \leqslant n$.
Рассмотрим (в данный момент равный $0$\footnote{Почему?}) определитель:
\begin{gather*}
    \det
    \begin{pmatrix}
        u_0(t_0) & … & u_0(t_n) \\
        u_1(t_0) & … & u_1(t_n) \\
        … & … & … \\
        u_n(t_0) & … & u_n(t_n) \\
    \end{pmatrix}=\\
    \det
    \begin{pmatrix}
        u_0(t_0) & … & u_0(t_i) & u_0(t_{i+1}) & \ldots & u_0(t_{i + q}) & u_0(t_{i + q + 1}) &\ldots & u_0(t_n) \\
        u_1(t_0) & … & u_1(t_i) & u_1(t_{i+1}) & \ldots & u_1(t_{i + q}) & u_1(t_{i + q + 1}) &\ldots & u_1(t_n) \\
        … & … & … & \ldots & \ldots & \ldots & \ldots & \ldots & \ldots\\
        u_n(t_0) & … & u_n(t_i) & u_n(t_{i+1}) & \ldots & u_n(t_{i + q}) & u_n(t_{i + q + 1}) &\ldots & u_n(t_n)
    \end{pmatrix}
\end{gather*}
Заменим столбцы с $i + 1$ до $i + q$ следующим образом:

\begin{gather*}
    \det
    \begin{pmatrix}
        u_0(t_0) & … & u_0(t) & u_0'(t) & \ldots & u_0^{(q)}(t) & u_0(t_{i + q + 1})& \ldots & u_0(t_n) \\
        u_1(t_0) & … & u_1(t) & u_1'(t) & \ldots & u_1^{(q)}(t) & u_1(t_{i + q + 1})& \ldots & u_1(t_n) \\
        … & … & … & \ldots & \ldots & \ldots & \ldots & \ldots & \ldots  \\
        u_n(t_0) & … & u_n(t) & u_n'(t) & \ldots & u_n^{(q)}(t) & u_n(t_{i + q + 1})& \ldots & u_n(t_n)
    \end{pmatrix}
\end{gather*}

Теперь должна быть понятна идея обобщения! Заменяем столбцы, с совпадающими точками на столбцы от соответствующих производных функций.

{\footnotesize
Перейдем теперь к строгому описанию. Пусть функции $\{u_i\}_{i=0}^n$, заданные на отрезке $[a, b]$, непрерывно дифференцируемы
$p$ раз на интервале $(a, b)$. Рассмотрим набор точек $t_0 \leqslant t_1 \leqslant \ldots \leqslant t_n \in [a, b]$ такой,
что 
\begin{align*}
    t_0 = \ldots = t_{k_0} < t_{k_0 + 1} = \ldots = t_{k_1} < t_{k_1 + 1} = \ldots = t_{k_2} < \ldots < t_{k_{\ell-1} + 1} 
    = \ldots = t_{k_\ell} = t_n,
\end{align*}
где $0 \leqslant \ell \leqslant n$\footnote{Чему соответствуют крайние случаи?} и
$0 \leqslant k_0 < k_1 < \ldots < k_\ell \leqslant n$\footnote{Можно ли здесь допустить нестрогое неравенство?}.

При этом дополнительно\footnote{Все три условия являются формальными. Тем не менее убедитесь, что понимаете, откуда они берутся.}
\begin{enumerate}
    \item $k_{i+1} - k_i \leqslant p - 1$ для всех $i$;
    \item если $t_{k_0} = a$, то $k_0 = 0$;
    \item если $t_{k_\ell} = b$, то $k_{\ell-1} = n-1$.
\end{enumerate}

Рассмотрим определитель
\setcounter{MaxMatrixCols}{20}
\begin{gather}
    \label{sec5::eq::U_star}
    U^\star \left(
    \begin{array}{cccc}
        0 & 1 & … & n \\
        t_0 & t_1 & … & t_n\\
    \end{array}  \right) = \\
    \nonumber
    \det
    \begin{pmatrix}
        u_0(t_{k_0}) & u_0'(t_{k_0}) & \ldots & u_0^{(k_0)}(t_{k_0}) & u_0(t_{k_1}) & u_0'(t_{k_1}) & \ldots & u_0^{(k_1-k_0)}(t_{k_1}) & \ldots & u_0(t_{k_\ell}) & \ldots & u_0^{(k_\ell-k_{\ell - 1})}(t_{k_\ell}) \\
        u_1(t_{k_0}) & u_1'(t_{k_0}) & \ldots & u_1^{(k_0)}(t_{k_0}) & u_1(t_{k_1}) & u_1'(t_{k_1}) & \ldots & u_1^{(k_1-k_0)}(t_{k_1}) & \ldots & u_1(t_{k_\ell}) & \ldots & u_1^{(k_\ell-k_{\ell - 1})}(t_{k_\ell}) \\
                   … &             … & \ldots &               \ldots &       \ldots &        \ldots & \ldots &                   \ldots & \ldots &          \ldots & \ldots & \ldots \\
        u_n(t_{k_0}) & u_n'(t_{k_0}) & \ldots & u_n^{(k_0)}(t_{k_0}) & u_n(t_{k_1}) & u_n'(t_{k_1}) & \ldots & u_n^{(k_1-k_0)}(t_{k_1}) & \ldots & u_n(t_{k_\ell}) & \ldots & u_n^{(k_\ell-k_{\ell - 1})}(t_{k_\ell}) 
    \end{pmatrix}
\end{gather}
А что будет, если все точки $\{t_i\}_{i=0}^n$ равны и $n = p$? Как называется такой определитель?
}

\begin{dfn}(Конструктивное)
    \label{def::ChebExtendedP}
    Система непрерывных функций $\{u_i\}_{i = 0}^n$, заданных на отрезке $[a, b]$, называется расширенной системой функций Чебышева порядка $p$, если
    функции $\{u_i\}_{i = 0}^n$ непрерывно дифференцируемы $p - 1$ раз и по всем наборам точек $\{t_i\}_{i=0}^n$, удовлетворяющим условиями сформулированным выше
\begin{gather}
    U^\star \left(
    \begin{array}{cccc}
        0 & 1 & … & n \\
        t_0 & t_1 & … & t_n\\
    \end{array}  \right) > 0.
\end{gather}
\end{dfn}

Если в Определении \ref{sec5::eq::U_star} взять $p = n + 1$, то такую систему функций $\{u_i\}_{i = 0}^n$ обычно называют расширенной системой Чебышева.

Как можно догадаться, есть и эквивалентное аксиоматическое определение расширенной Чебышевской системы.

\begin{dfn}(Аксиоматическое)
    \label{def::ChebExtendedP2}
    Система непрерывных функций $\{u_i\}_{i = 0}^n$, заданных на отрезке $[a, b]$, называется расширенной системой функций Чебышева порядка $p$, если
    функции $\{u_i\}_{i = 0}^n$ непрерывно дифференцируемы $p - 1$ раз и произвольный обобщенный многочлен, построенный по системе $\{u_i\}_{i = 0}^n$,
    имеет не более $n$ нулей с учетом кратности.
\end{dfn}

Как и для обычных систем Чебышева нетрудно провести доказательство эквивалентности этих определений. Проводится это точно так же, как и в Вопросе 3.
\footnote{\color{blue} Поручить кому-то набрать это аккуратно.}

{\color{blue} Тут еще идет кусок про теорему Элвинга, но я там ничего не понял. Видимо это и есть суть применения этого билета. Надо спросить В.Б. о том, что сюда еще нужно написать.}
