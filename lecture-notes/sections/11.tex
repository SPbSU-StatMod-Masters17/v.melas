\section{Теорема о числе точек в локально-оптимальном плане}

Для $k=1,2$ мы явно построили локально-оптимальные планы. Для $k\geq 3$ верна следующая теорема\footnote{Мы же вроде получили, что для чебышевских систем мы получили, что точек в предельном плане будет $\leq \frac{p+1}{2}$, где $p$ — количество функций в чебышевской системе. Почему тут такой слабый результат.}:

\begin{thm}[Без доказательства]
При $k \geq 3$ число точек в оптимальном плане не превосходит $\frac{3k(k+1)}{4}$.
\end{thm}
 
% \begin{proof}
%     Кажется это следствие теоремы Каратеодори и не отличается от 7 вопроса.
% \end{proof}
