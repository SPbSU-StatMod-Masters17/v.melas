\section{Теорема о числе точек локально-оптимальном плане}

Для $k=1,2$ мы явно построили локально-оптимальные планы. Для $k\geq 3$ верна следующая теорема\footnote{Мы же вроде получили, что для чебышевских систем мы получили, что точек в предельном плане будет $\leq \frac{p+1}{2}$, где $p$ — количество функций в чебышевской системе. Почему тут такой слабый результат.}:

\begin{thm}
При $k \geq 3$ с число точек в оптимальном плане не превосходит $\frac{k(k+1)}{2}+1$.
\end{thm}
 
 \begin{proof}
     Кажется это следствие теоремы Каратеодори и не отличается от 7 вопроса.{\color{blue} TODO}. 
 \end{proof}
