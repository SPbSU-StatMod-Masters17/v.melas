\section{Теорема о числе опорных точек локально-оптимальных планов для Чебышевских систем}

Пусть у нас есть некоторая система Чебышева $\{u_i(t) \}_{i=0}^{p}$ на отрезке $[a,b]$.
В качестве множества $\mathcal X$\footnote{см. вопрос 1} выберем отрезок $[a, b]$.
Рассмотрим множество всех возможных приближенных (непрерывных) планов $\Xi$, определенное следующим образом\footnote{При
изучении Чебышевских систем мы предполагаем, что регрессия зависит только от одного признака. В этом вопросе вместо $x$ мы будем обозначать его за $t$}:

\begin{align}
\Xi_k &= \left\{\begin{pmatrix}
        t_1 & … & t_k \\
        \nu_1 & … & \nu_k \\
\end{pmatrix}\right\}_{\substack{t_i \in [a,b]  \\ i \in 1:k} \sum \nu_j = 1}; \\
\Xi &=  \bigcup \Xi_k.
\end{align}

Таким образом, мы на самом деле задали всевозможные конечные дискретные вероятностные меры на отрезке $[a, b]$.

{ \small Отступление.
Вспоминаем, что информационная матрица плана выглядит следующим образом\footnote{Мы считаем, что есть некоторое приближение истинного значения параметра, в котором рассматриваем все объекты.
Подробное изложение см. в вопросе 2.}:
\begin{equation*}
M(\xi) = \int f(t)\Tr{f(t)}d\xi(t), \text{ где } \xi \in \Xi.
\end{equation*}

При этом $f$ — это частные производные функции $\eta(t, \theta)$ по $\theta$. Достаточно часто эти производные образуют систему Чебышева, а это сильно упрощает жизнь и позволяет получать различные хорошие аналитические результаты. Собственно матрица $M(\xi)$ является вектором в $\R^{\frac{m(m+1)}{2}}$, а ее элементы имеют вид:
$$M_{ij} = \int u_{ij}(t)d\xi(t), \xi \in \Xi.$$ 

Cуществует такое $n$, что любую $M(\xi)$ можно представать, как 
$$ M(\xi) = \int u d\xi(t), \xi \in \bigcup\limits_{i=1}^{n}\Xi_i, \text{ где } u=(u_0, …, u_p)$$
Докажем этот факт. 
}

\begin{dfn}
Моментным пространством $\mathcal{M}_{p+1}$ по отношению к $\{u_i\}_{i=0}^{p}$ называется множество 
$$\mathcal{M}_{p+1} = \{ \lambda c, c = (c_0, …, c_p) \in \R^{p+1}, \lambda \ge 0 | c_i = \int u_i(t)d\xi(t), \xi \in \Xi \} \subset \R^{p + 1}.$$
\end{dfn}

Во-первых, заметим, что множестве $\mathcal M_{p + 1}$ является конусом.\footnote{Проведите формальную проверку, это тривиально (заодно вспомните определение конуса).}
Во-вторых, обратим внимание, что это множество не образует $p$-мерную гиперплоскость в пространстве $\R^{p + 1}$.
Действительно, рассмотрим набор из $n + 1$ различной точки: $t_0, t_1, \ldots, t_n \in [a, b]$ и $c^{(i)} = \Tr{(u_0(t_i), u_1(t_i), \ldots u_n(t_i))} \in \R^{p + 1}$ для $i \in 0:n$.
Очевидно, что вектора $\{c^{(i)}\}_{i=0}^n$ являются линейно независимыми\footnote{Правда? Подсказка: $\{u_i\}$ образуют систему Чебышева.}. 
При этом, конечно, $c^{(i)} \in \mathcal M_{p + 1}$.\footnote{Да? А почему этого достаточно для доказательства?}

Сечение этого конуса $\lambda = 1$ — это в точности всевозможные информационные матрицы планов.

На самом деле верно следующее утверждение:
\begin{thm}
    $\mathcal M_{p + 1}$ --- замкнутый, выпуклый конус.
\end{thm}
В части замкнутости мы оставим это утверждение без доказательства. Однако, выпуклость легко проверить\footnote{И каждому следует это сделать.}.

Наша текущая цель: найти другие формы для представления структуры конуса $\mathcal M_{p + 1}$. Для этого мы введем несколько обозначений.
Введем кривую
$$C_{p+1} = \{ \gamma_t = (u_0(t), …, u_{p}(t)) | a\leq t \leq b\}.$$
Пусть $\mathcal{C}$ — наименьший выпуклый конус, содержащий кривую $C_{p+1}$. Рассмотрим множество $\Gamma$:
$$ \Gamma = \left\{ \gamma = \Tr{(\gamma_0, …, \gamma_p)}, \, \gamma_i = \sum \limits_{j=1}^{p+2} \lambda_j u_i(t_j) \ \bigg | \ \lambda_j \geq 0, a \leq t_j \leq b \right\} \subset \R^{p + 1}.$$

Так же нам потребуется следующая хорошо известная теорема
\begin{thm}[Каратеодори, без доказательства]
    \label{karateodoriThm}
    Пусть $\mathcal{V} \subset R^{k}$ — ограниченное замкнутое множество. Тогда любой элемент его выпуклой оболочки может быть представлен в виде линейной комбинации не более, чем $k+1$ элементов этого множества.
\end{thm}

Оказывается, что верно следующее.
\begin{thm}
    $\mathcal M_{p + 1} = \Gamma = \mathcal C$.
\end{thm} 
\begin{proof}
    Будем по порядку в этом убеждаться.
    Для начала ясно, что $\Gamma \subset \mathcal C$.\footnote{Прежде чем читать дальше, попробуйте доказать это сами.}
    Любой элемент $\gamma \in \Gamma$ имеет вид $$\gamma = \sum_{j = 1}^{p + 2} \lambda_j \Tr{(u_0(t_j), u_1(t_j), \ldots, u_p(t_j))},$$ где
    $\Tr{(u_0(t_j), u_1(t_j), \ldots, u_p(t_j))} \in C_{p + 1}$. Причем в силу того, что $\mathcal C$ выпуклое множество
    $$\sum_{j=1}^{n + 2} \frac{\lambda_j}{\sum_k \lambda_k} \Tr{(u_0(t_j), u_1(t_j), \ldots, u_p(t_j))} \in \mathcal C.$$
    Но $\mathcal C$ является конусом, а значит можно домножить на $\sum_k \lambda_k$.

    %Это множество совпадает с $\mathcal{C}$. То, что $\Gamma \subset \mathcal{C}$ очевидно.
    То, что $\Gamma \supset \mathcal C$ следует из теоремы Каратеодори\footnote{Хотя доказательство самой теоремы Каратеодори опускается, ее применение к нашей задаче должно быть очевидным!}.
    Таким образом, $\mathcal C = \Gamma$.
    Докажем теперь, что $\mathcal{C} = \mathcal{M}_{p+1}$. По построению ясно, что $\mathcal{C} \subset \mathcal{M}_{p+1}$.\footnote{Для формальной проверки воспользуйтесь знанием, что $\mathcal C = \Gamma$.
    Это простое упражнение на запись интерала по дискретной мере по определению.}
    Пусть теперь некоторый $\tilde{c} \in \mathcal{M}_{p+1}$, но $\tilde{c} \notin \mathcal{C}$. $\mathcal{C}$ является выпуклым замкнутым
    %\footnote{
    %Это вообще-то как-то не очевидно, а мы не доказывали. В книге Карлина используются неизвестные мне теоремы для док-ва…}
    конусом, поэтому по теоремам отделимости\footnote{Точнее нас интересует Вторая Теорема Отделимости. Можно освежить знания курса ФА (конец первого семестра). И да... Одноточечное множество в евклидовом пространстве,
    конечно, компактно (а почему? и почему я спросил?).}
    существует гиперплоскость, строго отделяющая $\tilde{c}$ от $\mathcal{C}$, т.е. существует такой вектор $a$ и $d \in \R$, что 
    \begin{gather*}
    \sum a_i \tilde{c}_i + d < 0 \\
    \sum a_i \gamma_i + d \geq 0 \text{ для всех }\gamma = \Tr{(\gamma_0, \ldots, \gamma_p)} \in \mathcal{C}
    \end{gather*}
    Из того, что $\gamma_i$ можно брать любым будет верно, что, в частности, 
    \begin{equation}
    \label{eqal}
    \sum a_i \lambda u_i(t) + d \geq 0 \text{ для всех } t \in [a,b].
    \end{equation}
    Из последнего неравенства следует, что $d\geq 0$, иначе при $\lambda = 0$ неравенство будет неверным. 
    Теперь рассмотрим дискретную меру $\sigma$, задающую $\tilde{c}$ (т.е. $\tilde{c} = \int u(t) d\sigma(t)$).  Пусть 
    $\lambda = \int d\sigma(t) > 0$. Тогда с одной стороны
    $$\sum a_i \tilde{c}_i + d = \int \sum a_i u_i d\sigma(t) + d  < 0$$
    С другой, подставив в \eqref{eqal} запись $\lambda$ через интеграл по мере мы получим противоречие.
    Доказательство завершенно.
\end{proof} 

%Теперь докажем теорему, которая, видимо, и имелась в виду в билете.
Теперь нам известно, что для любого $c \in \mathcal M_{p + 1}$ справедливо представление в виде, соответствующем определению $\Gamma$.
Вообще говоря, таких представлений много. На самом деле количество ненулевый слагаемых в определении $c$ можно сильно сократить, еще и обзавестись
единственностью разложения. Мы докажем это лишь в частном случае, когда точка находится на границе. На самом деле верен и более общий факт с небольшими уточнениями.

Предварительно требуется ввести следующее определение.
\begin{dfn}
    Индексом точки $c \in \mathcal{M}_{p+1}$, обозначаемым $I(c)$, называется такое минимальное $k$, что $c$ представима в виде выпуклой комбинации элементов $C_{p+1}$:
\begin{equation}
c = \sum \limits_{i=1}^{k} \lambda_i u(t_i)
\end{equation}
При этом точки $a$ и $b$ считаются за половину, а точки из $(a,b)$ за единицу.
\end{dfn}

\begin{thm}
    Ненулевой $\tilde{c} \in \mathcal{M}_{p+1}$ является граничной точкой тогда и только тогда, когда $I(\tilde{c}) \le (p+1)/2$\footnote{\color{blue} ANDY UPD.}. Кроме того, граничная точка $\tilde{c}$ допускает единственное представление
    $$\tilde{c} = \sum\limits_{i=1}^{k}\lambda_i u(t_i), \text{ где } k \leq \frac{p+2}{2}, \, \lambda_i > 0$$
    \footnote{\color{blue} Тут тоже UPD. Нужно проверять.}
\end{thm}

Пусть ненулевая точка $\tilde c$ является граничной точкой множества $\mathcal M_{p + 1}$. Известно, что тогда существует опорная гиперплоскость к $\mathcal M_{p + 1}$ в точке $\tilde c$,
которая задается вектором $\{a_i\}_{i = 0}^d$ и $d$, причем $\sum_{i=0}^p a_i^2 > 0$.\footnote{\color{blue} Этот момент я сам точно не понимаю.}
Причем как всегда
$$ \sum_{i=0}^p a_i c_i + d \ge 0,$$ где $c \in \mathcal M_{n + 1}$ и $\sum_{i=0}^n a_i \tilde c_i + d = 0$.
Покажем, что $d = 0$. Во-первых, ясно, что $d \ge 0$, так как $0 \in \mathcal M_{n + 1}$. Если же $d > 0$, то $\sum a_i \tilde c_i < 0$
и тогда для достаточно больших $\lambda$ будет справедливо неравенство $\lambda \sum a_i \tilde c_i + d < 0$\footnote{Здесь мы воспользовались тем, что $\mathcal M_{p + 1}$ --- конус}. Противоречие.

А значит
\begin{gather*}
    \sum_{i=0}^p a_i c_i \ge 0, \text{ при } c = \Tr{(c_0, c_1, \ldots, c_n)} \in \mathcal M_{p + 1}; \\
    \sum_{i=0}^p a_i \tilde c_i = 0.
\end{gather*}

Теперь определим $u^{(0)}(t) = \sum_{i=0}^n a_i u_i(t)$. Так как $C_{p + 1} \subset \mathcal M_{p + 1}$, то\footnote{Да?} $u^{(0)}(t) \ge 0$ для $a \le t \le b$.
Так как $\tilde c \in \mathcal M_{p + 1}$, то существует конечная дискретная мера $\xi \in \Xi$, такая что $\tilde c_i = \int u_i(t)\, d\xi(t)$.
Тогда $0 = \sum_{i=0}^p a_i \tilde c_i = \int u^{(0)}(t)\,  d\xi(t)$.
Так как $u^{(0)}(t) \ge 0$, то имеем $2 I(\tilde c) = \overline Z(u^{(0)})$.\footnote{см. билет 6.}
То есть $I(\tilde c) \le (p + 1) / 2$.
{ \bf Andy: TODO допонять конец теоремы и оформить кусочек после последнего gather по-человечески}.
{\color{blue} Я так и не понял, зачем мы рассматриваем конус (разве что из-за того, что так написано в книжке про чебышевские системы.) Теоремы отделимости работают и для выпуклых множеств, теорема Каратеодори сформулирована для выпуклой оболочки. Информационные матрицы для планов экспериментов также используют вероятностную меру. И вообще не уверен, что написанный текст соответствует вопросу…}
