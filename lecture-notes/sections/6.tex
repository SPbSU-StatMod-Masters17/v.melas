\section{Неотрицательные многочлены с заданными нулями}
\subsection{Теорема о числе нулей}
\begin{dfn}
Пусть $u$ — некоторая вещественная функция (непрерывная) на $[a,b]$.
Тогда обозначим $Z(u)$ число нулей $u$ на $[a,b]$.
\end{dfn}

\begin{dfn}
Ноль называется узловым, если выполнено одно из следующих условий
\begin{itemize}
\item Он совпадает с граничной точкой (либо $a$, либо $b$)
\item Функция меняет знак, проходя через этот ноль
\end{itemize}
В противном случае ноль называется неузловым.
\end{dfn}

\begin{dfn}
$\overline{Z(u)}$ — число нулей функции $u$, где неузловые нули засчитываются дважды.
\end{dfn}

\begin{thm}
\label{zeroThm}
Если система функций $\left\{ u_i\right\}_{i=0}^{n}$ - Чебышевская на $[a,b]$,  то для любого нетривиального многочлена $\overline{Z}(u) \leq n$.
\end{thm}

 \begin{proof}
Пусть $\overline{Z}(u) \geq n+1$ для некоторого нетривиального $u$. Обозначим различные нули $u$ через $t_1, …, t_k$. 
Добавим для первого неузлового нуля точки $t_i-\varepsilon, t_i+\varepsilon$, а для остальных неузловых нулей точки $t_i - \varepsilon$. Выбрав $\varepsilon$ достаточно маленьким, можно получить, что все точки будут содержаться в $[a,b]$. Пусть у нас было $m_1$ узловых и $m_2$ неузловых нулей. Тогда после проделанной операции мы получили $m_1 + 2m_2 + 1 \geq n+2$ точек ($m_1+2m_2 \geq n+1$).\footnote{Кроме случая, если $m_2$ равно $0$, но тогда
$Z(u) = \overline Z(u)$ и тогда условие Теоремы сводится к эквивалентности двух определений системы Чебышева, см. Теорему \ref{chebDefEqual}.}
Переобозначим получившиеся точки за $s_i$ и возьмем первые $n+2$ из них. Не умаляя общности, можем считать, что $u(s_{i}) \geq 0$ для четных $i$, $u(s_{i}) \leq 0$ для нечетных $i$\footnote{Проверьте это. Достаточно нарисовать рисунок и все станет ясно.}. Отсюда получаем, что следующий определитель равен 0 (т.к. первая стручка — линейная комбинация следующих):
\begin{equation}
\label{proofDet}
\left| \begin{array}{cccc}
u(s_0) & u(s_1) & … & u(s_{n+1}) \\
u_0(s_0) & u_0(s_1) & … & u_0(s_{n+1})\\
… & … & … & …\\
u_n(s_0) & u_n(s_1) & … & u_n(s_{n+1})\\ 
\end{array}
\right| = 0
\end{equation}

%Далее $\{u_i\}$ — система Чебышева, а значит 
%$$ \left| \begin{array}{cccc}
%u_0(t_0) & u_0(s_1) & … & u_0(t_{n})\\
%… & … & … & …\\
%u_n(t_0) & u_n(t_1) & … & u_n(t_{n})\\ 
%\end{array}
%\right| > 0
%$$
%для любых $t_0 < t_1 < … < t_n$. 
Разложив определитель \eqref{proofDet} по первой сточке, получим\footnote{как мы все помним, при разложение определителя знаки перед минорами чередуются, а сами миноры у нас положительны, см. \ref{chebSystem1}}, что 
$$\sum\limits_{i=0}^{n+1} \alpha_iu(s_i) = 0,$$
где $\alpha_i$ строго чередуются в знаке. Кроме того, $u(s_i)$ совпадают по знаку с $\alpha_i$. Таким образом, суммируются неотрицательные слагаемые. Значит $\forall i u(s_i) = 0$\footnote{Поскольку $\alpha_i = 0$ для всех $i$.}.
    Получили противоречие с одним из определений системы Чебышева\eqref{chebSystem2}.
 \end{proof}

Справедливо и обратное утверждение.

\begin{thm}
    Если для любого нетривиального многочлена $u(t)$ построенного по системе вещественных непрерывных $\{u_i\}_{i = 0}^n$
    верно, что $\overline{Z}(u) \leq n$, то система является Чебышевской.\footnote{С точностью до знака одной из функций $u_i$.
    Это понятно, если посмотреть на определение Чебышевской системы через определители.}
\end{thm}
\begin{proof}
Следует из второго определения чебышевской системы\footnote{$Z(u)$ ведь количество нулей многочлена}, см. \ref{chebSystem2}:
$$Z(u) \leq \overline{Z}(u) \leq n$$.
\end{proof}

\subsection{Неотрицательные многочлены с заданными нулями}
Задача: построить неотрицательный многочлен, имеющий нули в точках $t_1 < t_2 < … < t_k$, $t_i \in [a, b]$. Многочлен неотрицательный, поэтому все внутренние нули должны быть неузловыми. Введем функцию $\omega$:
$$ \omega(t) = \begin{cases}
    1, \text{ при }t = a \text{ или }t = b\\
    2, \text{ при }t \in (a, b).
\end{cases}
$$
\begin{thm}
    Пусть $t_1 < t_2 <  … < t_k$, $t_i \in [a; b]$, такие, что $\sum \limits_{i=1}^{k} \omega(t_i) \leq n$\footnote{Это условие из предыдущей теоремы.
    Можно сказать, что это условие на $n$.}. Пусть $\{u_i\}_{i=0}^{n}$ — чебышевская. Тогда $\exists u(t)$, который обращается в ноль в этих и только этих точках, за исключением случая, когда $n=2m$ и одна из точек совпадает с граничной точкой\footnote{Исключение получается по следующей простой причине: до этого мы доказали теорему о том, что число нулей $\overline{Z} \leq n$. Если $n=2m$, и одна точка совпадает с граничной, то $k < m$ и $2k+1 < 2m$, а значит возможна ситуация, что во второй граничной точке также будет ноль.}
\end{thm}

\textbf{В книжке было дополнительное условие, кажется без него док-во ломается…}
\textit{Andy: Как я понимаю, это только если мы хотим еще и <<исключительный случай>> обойти, но нам видимо это не нужно.}

\begin{proof}
Докажем для $n=2m+1$ и $a < t_1 < … < t_k < b$\footnote{Остальные случаи получаются аналогично с небольшими модификациями.}. Построим 
последовательность точек $\{s_i\}_{i=0}^{2m+1}$ следующим образом: добавим к $t_1,…, t_k$ произвольные точки $t_{k+1}, …, t_{m}$ такие, что $t_{k+1} < … < t_{m} < b$,  а затем добавим точки $t_1+\varepsilon, t_2 + \varepsilon$ и точку $a$. Получим $2m+1$ точки:
$$ s_0 = a, s_1=t_1 < s_2 = t_1 + \varepsilon < s_3 = t_2 < … < s_{2m+1} = t_{m}+\varepsilon$$
Теперь рассмотрим многочлен 
$$u_{\varepsilon}(t) = 
    U\begin{pmatrix}
        0 & 1 & … & 2m & 2m+1 \\
        s_0 & s_1 & … & s_{2m}& t\\
    \end{pmatrix}\footnotemark.
$$
\footnotetext{Здесь написан определитель матрицы (смотри определение \eqref{chebSystem1})}
Заметим, что по свойству определителя $u_{\varepsilon}(t)$ обращется в ноль в точках $s_0, s_1, …, s_{2m}$\footnote{Получаются одинаковые строчки
или столбцы, правда?}.
Кроме того, $\{u_i\}$ — система Чебышева, поэтому других нулей быть не может\footnote{Посмотрите, сколько корней у $u$? А сколько может быть максимально?}, 
а также каждый нуль является узловым (из теоремы \eqref{zeroThm}). Теперь, если $t > s_{2m+1}$, то из первого определения системы Чебышева \eqref{chebSystem1} следует, что $u_{\varepsilon}(t) > 0$. Следовательно $U_\varepsilon(t)$ на интервалах $(t_i, t_i+\varepsilon)$ будет меньше нуля, а на оставшихся — больше. Раскроем определитель по последнему столбцу и получим:
$$ u_\varepsilon(t) = \sum\limits_{i=0}^{2m} a_i(\varepsilon)u_i(t)$$
Можно считать, что $\sum a_i^2 = 1$ (если не так – нормируем). Тогда определим предельный многочлен $\overline{u}(t) = \lim \limits_{\varepsilon \rightarrow 0} u_\varepsilon(t)$.
{\color{blue} Было бы неплохо обосновать переход к пределу. Но сейчас нет времени.}
Теперь, у предельного многочлена нули в точках $t_1, …, t_k$ стали неузловыми, так как этот многочлен получился неотрицательным,
а значит данные точки являются всеми возможными нулями данного многочлена (опять же, по теореме \eqref{zeroThm}). Полученный многочлен имеет «лишний» ноль — в точке $a$. Чтобы от него избавиться, повторим построение, взяв вместо точку $b$ вместо точки $a$ и получим $\tilde{u}(t)$. Тогда решением нашей задачи будет многочлен
$u(t) = \overline{u}(t) + \tilde{u}(t)$.
\end{proof}
