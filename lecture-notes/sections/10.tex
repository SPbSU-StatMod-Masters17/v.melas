% \section{Экспоненциальные модели. Предельные планы.}

% В этом вопросе предлагается алгоритм для нахождения локально-$D$-оптимального плана в рамках модели экспоненциальной регрессии.
% Напомним, что локально-$D$-оптимальный план максимизирует определитель информационной матрицы $M(\xi, \Theta)$.
% Пусть у нас задано некое $\Theta_0 = \{a_i, b_i\}_{i=1}^k$, где $k$ --- количество слагаемых в модели экспоненциальной регрессии, а $a_i$ можно считать равными некой удобной константе,
% поскольку они не влияют на оптимизацию определителя.\footnote{Обо всем этом уже шла речь, здесь мы не будем повторяться.}

% Пусть в начале ($0$-вой шаг алгоритма) задано $2k + 1$ точка: $a = x_1^{(0)} < x_2^{(0)} < \ldots < x_{2k}^{(0)} < x_{2k + 1}^{(0)} = b$.
% Опишим $s$-тый шаг алгоритма.
% Определим $x_{2k + 1}^{(s)} = x_{2k + 1}^{(s-1)}$ и последовательно для всех $j = 2k, \ldots, 2$ положим
% $$x_{j}^{(s)} = \argmax_{z \in \left[x_{j - 1}^{(s-1)}, x_{j+1}^{(s)}\right]} M\left(\Tr{\left(x_1^{(s-1)}, x_2^{(s-1)}, \ldots, x_{j - 1}^{(s-1)}, z, x_{j + 1}^{(s)}, \ldots, x_{2k}^{(s)}\right)}, \Theta_0\right).$$

% Утверждается, что для всех $j$ и $s$ внутри отрезка $\left[x_{j - 1}^{(s-1)}, x_{j+1}^{(s)}\right]$ существует единственный локальный максимум функции $\det(M(\cdot, \Theta_0))$.\footnote{\color{blue} Непонятно.}
% Также утверждается, что этот метод всегда сходится, то есть можно выделить подпоследовательность $\{ \Tr{(a, x_2^{s_i}, x_3^{s_i}, \ldots, x_{2k}^{s_i})} \}_{i = 1}^\infty$, которая сходится к
% стационарной точке функции $\det(M(\cdot, \Theta_0))$. Точнее $x_j^{(s_i)} \to x_j^*$ при $i \to \infty$ и $j = 2, \ldots, 2k$.
% {\color{blue} Как в точности проходит доказательство, не совсем понятно. Кажется, мы предполагаем, что какая-то последовательность не является фундаментальной и приходим к противоречию.}
