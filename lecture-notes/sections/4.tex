
\section{Системы Чебышева. Метод проверки, основанный на последовательном дифференцировании. Примеры применения (экспоненциальные модели.)}

Пусть $u_0, u_2,…, u_k$ — некоторая система функций. Мы хотим проверить, что она является Чебышевской.
Предлагается метод, предлагающий конструктивную возможность определить является ли набор функций Чебышевской системой,
основанный на дифференцировании функций $u_i$. Для этого будем предполагать, что они обладают достаточной гладкостью в $(a, b)$.
Рассмотрим следующий набор функций:
$$ F_{00}(t) = u_0(t), …, F_{0k}(t) = u_k(t)$$
$$F_{11}(t) = \left(\frac{F_{01}}{F_{00}}\right)^{'}, …, F_{1k}(t) = \left(\frac{F_{0k}}{F_{00}}\right)^{'}$$
$$F_{22}(t) = \left(\frac{F_{12}}{F_{11}}\right)^{'}, …, F_{2k}(t) = \left(\frac{F_{1k}}{F_{11}}\right)^{'} $$
$$ … $$
$$F_{kk} = \left(\frac{F_{k-1, k}}{F_{k-1, k-1}}\right)^{'}$$
\begin{thm}
\label{seqDerTh}
Если существуют все функции $F_{ij}$ при $0 \le j \le i \le k$ и $F_{ii} > 0$, то система $u_0, …, u_k$ является системой Чебышева.
\end{thm}
\begin{proof}
    Пусть это не так. Тогда $\exists u(t) = \sum \limits_{i=0}^{k} a_i u_i$, обращающийся в 0 в $k+1$ точках. Не умаляя общности будем считать, что все $a_i \neq 0$\footnote{Иначе
    проводим доказательство по той же схеме, но с $k$ равным числу ненулевых $a_i$.}.
Тогда
$$ f_0(t) = a_0u_0(t)\left(1 + \frac{a_1}{a_0} \frac{u_1(t)}{u_0(t)} + … \frac{a_k}{a_0}\frac{u_k(t)}{u_0(t)}\right).$$
По условию, $u_0(t) = F_{00}(t) > 0$, а значит вторая скобка обращатся в 0 в $k+1$ точках. Вспоминаем теорему Ролля — между двумя корнями непрерывной и дифференцируемой функции есть корень ее производной. Отсюда следует, что функция 
$f_1(t) = \left(1 + \frac{a_1}{a_0} \frac{u_1(t)}{u_0(t)} + … \frac{a_k}{a_0}\frac{u_k(t)}{u_0(t)}\right)^{'}$ — обращется в ноль в $k$ точках. Заметим, что количество слагаемых уменьшилось на 1.
Применяем приведенное рассуждение к $f_1(t)$, получаем функцию $f_2(t)$ и т.д. В итоге получим 
последовательность функций $f_0(t), f_1(t), …, f_k(t)$.
В $f_i(t)$ будет $k-i+1$ ненулевое слагаемое и $k-i+1$ нуль.
Таким образом, $f_k(t) = \alpha F_{kk}$, где $\alpha$ — некоторое ненулевое число, имеет хотя бы один ноль. Противоречие, т.к. по предположению $F_{kk}(t) > 0$.
\end{proof}

\subsection{Пример: Экспоненциальная регрессия}
Пусть $\theta = \Tr{(\lambda_0, \ldots, \lambda_k, b_0, \ldots, b_k)}$, где $b_i \in \R, \lambda_i \in \R, \lambda_i \neq \lambda_j i \neq j$.
Модель задается функцией\footnote{Напомним, что аналитический вид регрессии предполагается известным.} 
$\eta(t, \theta) = \sum\limits_{i=0}^k b_i e^{\lambda_it}$.
Рассмотрим систему функций 
$\left\{ \frac{\partial \eta(t, \theta)}{\partial \lambda_i}, \frac{\partial \eta(t, \theta)}{\partial b_i} \right\}_{i=0}^{k}$.
Оказывается, данная система является системой Чебышева.
Чтобы это показать легче всего повторить рассуждение, легшее в основу доказательства Теоремы \eqref{seqDerTh} и воспользоваться тем, что $e^{\lambda t} > 0$ для всех $\lambda \in \R$.

\subsection{Пример: модель Михаэлина-Менте}\footnote{я наверно не правильно распарсил имена, надо поправить}
$\eta(t, \theta) = \frac{at}{t+b}$ на $[c, d]$, $a > 0$, $\theta = \Tr{(\theta_1, \theta_2)} = \Tr{(a, b)}$.
Производные $\left\{\frac{\partial\eta}{\theta_i}\right\}$ также образуют систему Чебышева. 

Это легко показать, применив Теорему \eqref{seqDerTh}.

{\footnotesize Тем не менее доказательство можно провести и непосредственно, используя схему доказательства Теоремы \eqref{seqDerTh}.
Действительно,
$$\frac{\partial\eta}{a} = u_0(t) = \frac{t}{t+b}$$
$$ \frac{\partial\eta}{b} = u_1(t) = \frac{-at}{(t+b)^2}$$

Пусть имеется $u(t) = \alpha_0 u_0(t) - \alpha_1u_1(t)$. Вынесем $-u_1(t)$ за скобку и получим
$$u(t) = \frac{at}{(t+b)^2}\left( \alpha_0 (t+b) + \alpha_1  \right)$$
Вспомним, что $a >0$, а значит $t > 0$ и $\frac{at}{(t+b)^2} > 0$. Второе слагаемое — линейная функция, про которую мы и без дифференцирования знаем, что у нее имеется не более одного нуля.}
