
\section{Системы Чебышева. Метод проверки, основанный на последовательном дифференцировании. Примеры применения (экспоненциальные модели.)}

Пусть $u_0, u_2,…, u_k$ — некоторая система функций. Мы хотим проверить, что она является Чебышевской. 
Рассмотрим следующий набор функций:
$$ F_{00}(t) = u_0(t), …, F_{0k}(t) = u_n(t)$$
$$F_{11}(t) = \left(\frac{F_{01}}{F_{00}}\right)^{'}, …, F_{1k}(t) = \left(\frac{F_{0k}}{F_{00}}\right)^{'}$$
$$F_{22}(t) = \left(\frac{F_{12}}{F_{11}}\right)^{'}, …, F_{2k}(t) = \left(\frac{F_{1k}}{F_{11}}\right)^{'} $$
$$ … $$
$$F_{kk} = \left(\frac{F_{k-1, k}}{F_{k-1, k-1}}\right)^{'}$$
\begin{thm}
\label{seqDerTh}
Если существуют все функции $F_{ij}$ и $F_{ii} > 0$, то система $u_0, …, u_k$ является системой Чебышева.
\end{thm}
\begin{proof}
Пусть это не так. Тогда $\exists u(t) = \sum \limits_{i=0}^{k} a_i u_i$, обращающийся в 0 в $k+1$ точках. Не умаляя общности будем считать, что все $a_i \neq 0$. Тогда 
$$ f_0(t) = a_0u_0(t)\left(1 + \frac{a_1}{a_0} \frac{u_1(t)}{u_0(t)} + … \frac{a_k}{a_0}\frac{u_k(t)}{u_0(t)}\right)$$
По условию, $u_0(t) > 0$, а значит вторая скобка обращатся в 0 в $k+1$ точках. Вспоминаем теорему Ролля — между двумя корнями непрерывной функции есть корень ее производной. Отсюда следует, что функция 
$f_1(t) = \left(1 + \frac{a_1}{a_0} \frac{u_1(t)}{u_0(t)} + … \frac{a_k}{a_0}\frac{u_k(t)}{u_0(t)}\right)^{'}$ — обращется в ноль в $k$ точках. Заметим, что количество слагаемых уменьшилось на 1. Итерируя процесс, получим 
последовательность функций $f_0(t), f_1(t), …, f_k(t)$. В $f_i(t)$ будет $k-i+1$ ненулевых слагаемых и $k-i$ нулей. Таким образом, $f_k(t) = \alpha F_{kk}$, где $\alpha$ — некоторое ненулевое число, имеет хотя бы один ноль. Противоречие, т.к. по предположению $F_{kk}(t) > 0$
\end{proof}
\subsection{Пример: Экспоненциальная регрессия}
Пусть $\eta(t, \theta) = \sum\limits_{i=1}^k b_i e^{\lambda_it}$, $b_i \in \R, \lambda_i \in \R, \lambda_i \neq \lambda_j i \neq j$. В данной модели параметрами являются $b_i$ и $\lambda_i$ и они входят нелинейно. Рассмотрим систему функций 
$\left\{ \frac{\partial \eta(t, \theta)}{\partial \lambda_i}, \frac{\partial \eta(t, \theta)}{\partial b_i} \right\}_{i=1}^{k}$. Оказывается, данная система является системой Чебышева. Для доказательства достаточно повторить рассуждение, легшее в основу доказательства прошлой теоремы \eqref{seqDerTh} и воспользоваться тем, что $e^{\lambda t} > 0 \forall \lambda \in R$.

\subsection{Пример: модель Михаэлина-Менте}\footnote{я наверно не правильно распарсил имена, надо поправить}
$\eta(t, \theta) = \frac{at}{t+b}$ на $[a, b]$, $a > 0$. 
Производные $\left\{\frac{\partial\theta}{\theta_i}\right\}$ также образуют систему Чебышева. 

Действительно,
$$\frac{\partial\eta}{a} = u_0(t) = \frac{t}{t+b}$$
$$ \frac{\partial\eta}{b} = u_1(t) = \frac{-at}{(t+b)^2}$$

Пусть имеется $u(t) = \alpha_0 u_0(t) - \alpha_1u_1(t)$. Вынесем $-u_1(t)$ за скобку и получим
$$u(t) = \frac{at}{(t+b)^2}\left( \alpha_0 (t+b) + \alpha_1  \right)$$
Вспомним, что $a >0$, а значит $t > 0$ и $\frac{at}{(t+b)^2} > 0$. Второе слагаемое — линейная функция, про которую мы и без дифференцирования знаем, что у нее имеется не более одного нуля.


