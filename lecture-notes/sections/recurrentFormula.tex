\section{Реккурентная формула для коэффициентов Тейлора неявных функций. Вывод и примеры использования.}

Пусть у нас есть некоторая непрерывно-дифференцируемая в окрестности $U \in \R^k \times \R^r$ функция $g(\tau, z)$, где $\tau \in \R^k$, $z \in \R^r$, для
которой выполнены условия теоремы о неявной функции. Точнее
\begin{gather*}
    g(\tau_0, z_0) = 0, \\
    \det J \neq 0,
\end{gather*}
где 
$$J(\tau, z) = \left(\frac{\partial}{\partial \tau_i} g_j(\tau, z)\right)_{i, j = 1}^m.$$

По теореме о неявной функции существует $\tau(z)$ такая, что $\tau(z_0) = \tau_0$ и $g(\tau(z), z) = 0$ в некоторой окрестности $z \in V$.
Также мы действуем в предположении, что $g$ является аналитической, а значит и $\tau$ будет являться аналитической по теореме о неявной функции.
Наша текущая цель научиться определять коэффициенты в разложении Тейлора функции $\tau(z)$ в окрестности точки $z_0$.
Начнем со случая $k = 1$.
$$ \tau(z) = \tau_{(0)} + \sum_{i=1}^\infty \tau_{(i)} z^i.$$
Положим $\tau_{(0)} = \tau_0$ для единообразия.
Обозначим $\tau_{<n>}(z) = \tau_{(0)} + \sum_{i=1}^n \tau_{(i)} z^i$ для всех $n \in \N_0$ и
$J_{(0)} \overset{\mathrm{def}}{=} J(\tau_0, z_0)$.
\begin{thm}[Рекуррентные формулы для коэффициентов Тейлора]
    $$\tau_{(n)} = -\frac{1}{n!} J_{(0)}^{-1} \frac{\partial g}{\partial z^n}\big(\tau_{<n - 1>}(z), z\big) \bigg |_{z = z_0}.$$
\end{thm}

\begin{proof}
TODO
\end{proof}

\begin{ex}
    Рассмотрим функцию $g(\tau, z) = \sin z - \ln \tau$, разложим ее в ряд Тейлора в окрестности точки $z_0 = 0$.
    $\tau_{(0)} = 1$, так как $g(\tau_0, z_0) = 0$.
    Ясно, что $J_{(0)} = -1$.
    Далее непосредственное вычисление показывает, что $\tau_{(1)} = 1$.
    Подробнее выпишем вычисление $\tau_{(2)}$.
    \begin{gather*}
        \tau_{(2)} = -\frac{1}{2} J_{(0)}^{-1} \big(g(\tau_{<1>}(z), z\big)' \big |_{z = z_0} = \frac{1}{2} (\sin z - \ln(1 + z))'' \big |_{z = z_0} = \\
        = \frac{1}{2} (-1) \left(\frac{1}{1 + z}\right)' = \frac{1}{2}.
    \end{gather*}
    Далее вычисления можно производить, сколь угодно долго.
\end{ex}

\begin{ex}
    Рассмотрим функцию $g(\tau, z) = \tau^3 + \tau^2 - 2\tau + z$, разложим ее в ряд Тейлора в окрестности точки $z_0 = 0$.
    $\tau_{(0)} = 1, -2, 0$, так как $g(\tau_0, z_0) = 0$.
    Выберем, например, $\tau_{(0)} = 0$.
    При этом $J(\tau, z) = 3 \tau^2 + 2\tau -2$, а значит $J_{(0)} = -2$.
    Далее непосредственное вычисление показывает, что $\tau_{(1)} = 1/2$.
\end{ex}

\bigskip
Здесь приводится изложение для многомерного случая, взятое целиком из пособия.

Пусть $m$ и $k$ --- произвольные натуральные числа,
    $\tau=(\tau_1,\ldots,\tau_m) \in {\R}^m$, $z=(z_1,\ldots,z_k)\in
{\R}^k$ и $g(\tau,z)=(g_1(\tau,z), \ldots,g_m(\tau,z))$ ---
вещественная аналитическая вектор-функция в некоторой окрестности
$U$ точки $(\tau_{(0)},z_{(0)})$, такая, что
$g(\tau_{(0)},z_{(0)})=0$.

Для любого набора индексов $s=(s_1,\ldots,s_k)$, $s_i\geq 0$,
    $i=1,\ldots,k$ и любой (скалярной, векторной или матричной)
    вещественной аналитической функции $f$ обозначим
    $$
    f_{(s)}=\left(f(z)\right)_{(s)}=\frac{1}{s_1!\ldots s_k!}
    \frac{\partial^{s_1}}{\partial z_1^{s_1}}\ldots
    \frac{\partial^{s_k}}{\partial z_k^{s_k}} f(z)|_{z=z_{(0)}}.
    $$

    Обозначим
    $$
    S_t=\{s=(s_1,\ldots,s_k);\quad s_i\geq 0,\quad \sum^k_{i=1} s_i=t\}
    $$
    при $t=0,1,\ldots$.

    Пусть $l=(l_1,\ldots,l_k)$,$l_i\geq 0$ --- заданный вектор целых
    чисел.

    Предположим, что функция $g$ имеет вид
    $$
    g(\tau,z)=(z_1-z_{1(0)})^{l_1}\ldots (z_k-z_{k(0)})^{l_k}
    \psi(z)\tilde g(\tau,z),
    $$
    где $\psi(z)$ --- однородный многочлен степени $p\geq 0$,
    $$
    \psi(z)=\sum_{s\in S_p}a_{(s)}(z_1-z_{1(0)})^{s_1}\ldots
    (z_k-z_{k(0)})^{s_k},
    $$
    такой, что $a_{(p,0,\ldots,0)}\ne 0$, причем $\det \tilde
    J(\tau_{(0)},z_{(0)})\ne 0$, где
    $$
    \tilde J(\tau,z)=\frac{\partial}{\partial\tau}\tilde g(\tau,z).
    $$

    Рассмотрим уравнение
    $$
    g(\tau,z)=0.
    $$
    При $z_i\ne z_j$, $(i\ne j)$, $z_i\ne z_{i(0)}$, $i,j=1,\ldots,k$ и
    при $\psi(z)$ таком, что $\psi(z)\ne 0$ при $z_i\ne z_j$, $i\ne j$,
    $z_i\ne z_{i(0)}$, $i,j=1,\ldots,k$, это уравнение эквивалентно
    уравнению
    $$
    \tilde g(\tau,z)=0.
    $$

    Легко проверить, что функция $\tilde g(\tau,z)$ является
    вещественной аналитической при $(\tau,z)\in U$.

    В силу теоремы о неявной функции (Бибиков, 1991, с.\,173) существует
    вектор-функция $\tau^*(z)$,\, которая\, является\,
    ве\-ще\-ст\-вен\-ной аналитической вектор-функцией в некоторой
    окрестности $H$ точки $\tau=\tau_{(0)}$, и такая, что
    $\tau^*(z_{(0)})=\tau_{(0)}$, $(\tau^*(z),z)\in U$ и $\tilde
    g(\tau^*(z),z)=0$, а значит и $ g(\tau^*(z),z)=0$ при $z\in H$. Эта
    вектор-функция в некоторой окрестности $H^*\subset H$ точки
    $z_{(0)}$ разлагается в (сходящийся) ряд Тейлора:
    $$
    \tau^*(z)=\tau_{(0)}+ \sum^\infty_{t=1}\sum_{s\in S_t}
    \tau^*_{(s)}(z_1- z_{1(0)})^{s_1}\ldots(z_k-z_{k(0)})^{s_k}.
    $$
    Отметим, что при $k=1$ множество $S_t$ состоит из одного элемента
    $s=s_t=t$.

    Построим рекуррентные формулы для вычисления коэффициентов
    $\tau^*_{(s)}$, $s\in S_t$, $t=1,2,\ldots$.

    Пусть $I$\,---\,произвольное\,множество\,индексов
    вида\,$s=(s_1,\ldots,s_k)$, $s_i\geq 0$, $i=1,\ldots,k$.

    Обозначим
    $$
    \tau_{<I>}(z)=\sum_{s\in I}\tau_{(s)}(z_1-z_{1(0)})^{s_1}\ldots
    (z_k- z_{k(0)})^{s_k},
    $$
    $$
    J(\tau,z)=\frac{\partial}{\partial \tau}g(\tau,z),\quad
    J_{(l)}=\left(J(\tau_{(0)},z)\right)_{(l)}.
    $$

    Пусть сначала $p=0$.

    Введем множество индексов
    $$
    I_n=U^n_{t=0}S_t.
    $$

    Тогда имеет место следующее утверждение.

    \textbf{ Предложение 2.5.1.} { При сформулированных выше условиях
        имеют место следующие формулы
            $$
            \tau^*_{(s)}=-J^{-1}_{(l)}g(\tau^*_{<I>}(z),z)_{(s+l)}
        $$
            при $s\in S_t$,  $I=I_{t-1}$, $t=1,2,\ldots$ }

            Доказательство этого и следующего предложения будет дано в конце
            раздела.

            С помощью предложения 2.5.1 можно построить рекуррентный алгоритм
            вычисления коэффициентов разложения функции $\tau^*(z)$ в ряд
            Тейлора при $p=0$ и произвольном $l=(l_1,\ldots,l_k)$, $l_i\geq 0$,
            $i=1,\ldots,k$.

            \textbf{ Алгоритм 1.} На шаге $t$ $(t=1,2,\ldots)$ вычисляем все
            коэффициенты с индексами из $S_t$ по формуле
            $$
            \tau_{(s)}=-J^{-1}_{(l)}\left(g(\tau_{<I_{t-1}>}(z),z\right)_{(s+l)}.
                    $$

                    Из Предложения 2.5.1 вытекает, что $\tau_{(s)}=\tau^*_{(s)}$, $s\in
                    S_t$, $t=1,2,\ldots$.

                    Рассмотрим теперь случай $p>0$. Определим множество индексов
                    $$
                    \hat S_t=\{s=(s_1,\ldots,s_k);s_i\geq
                    0,i=1,\ldots,k,s_1+2\sum^k_{i=2} s_i=t\}.
                    $$

                    Пусть $\hat I_n= U^n_{t=0}\hat S_t$, $q=(p,0,\ldots,0)$.

                    Пусть $J_{(l+q)}=\left(J(\tau_{(0)},z)\right)_{(l+q)}$.


                    \textbf{ Предложение 2.5.2.} \textit{ При $p>0$ имеют место формулы
                    $$
                    \tau^*_{(s)}=-J^{-1}_{(l+q)}
                    g\left(\tau^*_{<I>}(z),z\right)_{(s+l+q)}
                    $$
                        при $s\in \hat S_t$, $I=\hat I_{t-1}$, $t=1,2,\ldots$. }

                        Алгоритм вычисления коэффициентов $\tau^*_{(s)}$ получаем из
                        Алгоритма 1 заменой множества $S_t$ на $\hat S_t$.


                        \textbf{ Алгоритм 2.} \textit{ При $p>0$ на шаге $t$ $(t=1,2,\ldots)$
                            вычисляем все коэффициенты с индексами $s\in\hat S_t$ по формуле
                                $$
                                \tau_{(s)}=-J^{-1}_{(l+q)} g\left(\tau_{<\hat
                                        I_{t-1}>}(z),z\right)_{(s+l+q)}.
                                $$
                        }

Из Предложения 2.5.2 вытекает, что $\tau_{(s)}=\tau^*_{(s)}$, $s\in
\hat S_t$, $t=1,2,\ldots,s$. Отметим, что при $k>1$ в Алгоритме 2 на
каждом шаге вычисляется приблизительно в $k$ раз меньшее число
коэффициентов, чем в Алгоритме 1, а при $k=1$ эти алгоритмы
совпадают.

Перейдем к доказательству предложений  2.5.1 и 2.5.2.

\textbf{ Доказательство предложения 2.5.1.} Пусть $\tau(z)$ ---
произвольная вещественная аналитическая в некоторой окрестности
точки $z_{(0)}$ вектор- функция. Рассмотрим следующий
вспомогательный результат.

\begin{lem}
 \label{th:ageom<cq}
 При сформулированных в начале раздела  условиях и при $p=0$, $l=0$
 справедливы равенства:
 $$
 \frac{\partial^n}{\partial z^n}\left[g\left(\tau_{(n)}(z),z\right)-
 g\left(\tau(z),z\right)\right]|_{z=z_{(0)}}=0,
 $$
 при $k=1$ и $\tau_{(n)}(z)=\sum^n_{i=1}\tau_{(i)}z^i+\tau_{(0)}$,
 $n=1,2,\ldots$,
 $$
 \frac{\partial^n}{\partial z^{s_1}_1\ldots\partial z^{s_k}_k} \left[
     g\left(\tau_{<I>}(z),z\right)-g\left(\tau(z),z\right)\right]|_{z=z_{(0)}}=0,
     $$
     при $k\geq 1$, $s\in S_n$, где $I=I_n$, $n=1,2,\ldots$.
     \end{lem}

     \textbf{ Доказательство.} Пусть сначала $k=1$.

     Заметим, что справедливо равенство
     $$
     \frac{\partial}{\partial z}
     g\left(\tau(z),z\right)=\frac{\partial}{\partial \tau}
     g(\tau,z)|_{\tau=\tau(z)} \times
     \tau^{'}(z)+\frac{\partial}{\partial z} g(\tau,z)|_{\tau=\tau(z)}.
     $$
     В самом деле, по определению производной и в силу формулы
     дифференцирования сложной функции, имеем \begin{align}
     &\frac{\partial}{\partial z} g(\tau(z),z)=\lim_{\Delta\to 0}
     \frac{g(\tau(z+\Delta),z+\Delta)-g(\tau(z),z)}{\Delta} = \nonumber\\
         &=\frac{\partial}{\partial \tau} g(\tau,z)|_{\tau=\tau(z)}
         \tau^{'}(z)+ \frac{\partial}{\partial z} g(\tau,z)|_{\tau=\tau(z)}.
         \nonumber\end{align}

         Используя это равенство, получим \begin{align} 
         &\frac{\partial^n}{\partial
             z^n} g(\tau(z),z)= \frac{\partial}{\partial \tau}g(\tau(z),z)
    \tau^{(n)}(z) +
    \frac{\partial^n}{\partial z^n} g(\tau,z)|_{\tau=\tau(z)}+\nonumber \\
        &+\sum^m_{i,j=1}\frac{\partial^2}{\partial\tau_i\partial\tau_j}
        g\left(\tau(z),z\right)\sum^{n-1}_{t=1}\tau^{(t)}_i(z)\tau^{(n-t)}_j(z)
    +\ldots+\nonumber\\
        &+\sum^m_{i_1,\ldots,i_n=1}
        \frac{\partial^n}{\partial\tau_{i_1}\ldots\partial\tau_{i_n}}
        g\left(\tau(z),z\right) \tau^{'}_{i_1}(z)\ldots
        \tau^{'}_{i_n}(z).\nonumber 
        \end{align}

        Переходя к пределу при $z\to z_{(0)}$, получаем \begin{align}
        &\frac{\partial^n}{\partial z^n} g(\tau(z),z)|_{z=z_{(0)}}= \nonumber \\
            &=n! J_{(0)}\tau_{(n)}+\frac{\partial^n}{\partial z^n}
            g(\tau_{(0)},
                    z_{(0)})+\ldots+  \\
                &+\sum^m_{i_1,\ldots,i_n=1}
                \frac{\partial^n}{\partial\tau_{i_1}\ldots\partial\tau_{i_n}}
                g(\tau_{(0)},z_{(0)})\tau_{i_1(1)}\ldots\tau_{i_n(1)} i_1!\ldots
                i_n!, \nonumber
                 \end{align} 
                где правая часть зависит только от
                $\tau_{(0)},\ldots,\tau_{(n)}$ и не зависит от
                $\tau_{(n+1)},\ldots$. Следовательно,
                $$
                \frac{\partial^n}{\partial z^n} g\left(\tau(z),z\right)|_{z=z_{(0)}}
                = \frac{\partial^n}{\partial z^n}
                g\left(\tau_{(n)}(z),z\right)|_{z=z_{(0)}}.
                $$

                Доказательство леммы для случая $k>1$ проводится аналогичным
                образом.

                Перейдем к доказательству \textbf{ Предложения 2.5.1}.
                \begin{proof} Пусть $k=1$,
                $l=0$. Заметим, что в правой части равенства (2.28) только первое
                слагаемое зависит от $\tau_{(n)}$, а остальные зависят только от
                $\tau_t$, $t\leq n-1$. Так как $g(\tau^*(z),z)\equiv 0$ при $z\in
                H$, то
                $$
                -\frac{\partial^n}{\partial z^n} g\left(\tau^*_{(n-1)}(z),
                        z\right)|_{z=z_0}= n! J_{(0)}\tau^*_{(n)}.
                $$
                При $k>1$, $l=0$, рассуждения проводятся аналогичным образом.

                В случае $l\ne 0$ формулы проверяются непосредственным вычислением
                производной с учетом леммы 2.5.1.
                \end{proof}
                Очевидно, что при $p=0$ формулы из предложения 1 остаются верными,
                если заменить в них $S_t$ на $\hat S_t$ и $I_t$ на $\hat I_t$.

                Рассмотрим теперь случай $p>0$.

                \textbf{ Доказательство Предложения 2.5.2.} \begin{proof}
                Пусть сначала $l=0$.
                Заметим, что \begin{equation} \left(g(\tau^*_{<I>}(z),z)\right)_{(s+q)}=
                \sum_{w+v=s+q}a_{(w)}\tilde g(\tau^*_{<I>}(z),z)_{(v)} \label{eq:2}
                \end{equation} для любого множества индексов $I$.

                При $w=q$, единственным вектором $v$, таким, что $w+v=s+q$ является
                вектор $s$. Пусть $s \in \hat S_n$, $I= \hat I_{n}$. Заметим, что
                при $w\ne q$ любой вектор $v$, такой что $w+v=s+q$, принадлежит
                множеству $\hat S_t$, $t\leq n-1$. Отсюда правая часть равенства
                (\ref{eq:2}) имеет вид
                $$
                a_{(q)}\tilde g\left(\tau^*_{<\tilde I_n>}(z),z\right)_{(s)}.
                $$
                Непосредственным вычислением можно проверить, что $J_{(q)}=a_{(q)}
                \tilde J_{(0)}$. Отсюда вытекает справедливость предложения 2.5.2
                при $l=0$. Для произвольного $l$ справедливость формул проверяется
                непосредственным вычислением.\end{proof}

                В работе (Мелас, Пепелышев, 1999) с помощью описанных здесь
                алгоритмов построены разложения в ряд Тейлора точек насыщенных
                локально $D$-оптимальных планов для функции регрессии вида
                $$
                \eta(x,\Theta)=\sum^k_{i=1}\theta_i {\mbox{exp}}\,(-\theta_{i+k}x),
                $$
                где $\Theta=(\theta_1,\ldots,\theta_{2k})^T$, $\theta_i\ne 0$,
                $\theta_{i+k}>0$ ($i=1,\ldots,k$), $x\in [0,\infty)$.


